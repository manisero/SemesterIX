\documentclass[a4paper,10pt]{article}
\usepackage[utf8]{inputenc}
\usepackage[MeX]{polski}
\usepackage{tikz}
\usepackage{url}
\usepackage{fullpage}
\usepackage{titlesec}
\usepackage{amsmath}

\title{Test pierwszości Solovaya-Strassena. \\ \small{Projekt z przedmiotu PTKB.}}
\author{Michał Aniserowicz}
\date{}

\begin{document}

\maketitle

\section{Opis zadania} \label{sec:task_desc}
Celem projektu jest zaimplementowanie probabilistycznego testu pierwszości Solovaya-Strassena.

\section{Teoria}
Probabilistyczny test pierwszości Solovaya-Strassena został opracowany przez Roberta M. Solovaya i Volkera Strassena.
Określa on, czy dana liczba jest liczbą złożoną czy prawdopodobnie pierwszą.

Podstawową wykorzystywaną przez niego własnością jest wykazany przez Eulera fakt, że dla każdej liczby pierwszej $p$ i dowolnej liczby naturalnej $a$, zachodzi:

$a^{(p-1)/2} \equiv \left(\frac{a}{p}\right) \pmod p$,

gdzie $\left(\frac{a}{p}\right)$ jest symbolem Legendre'a.

 \subsection{Symbol Legendre'a}
 Symbol Legendre'a to funkcja $\left( \frac a p \right)$ zdefiniowana następująco:

 $
 \left( \frac a p \right) = 
 \begin{cases}
   0 & \mbox{, jeśli $a \equiv 0 \pmod p$} \\
   1 & \mbox{, jeśli istnieje takie $b$, że $b^2=a \mod p$} \\
  -1 & \mbox{, jeśli nie istnieje żadne $b$ takie że $b^2=a \mod p$}
 \end{cases}
 $,

 gdzie $p$ jest liczbą pierwszą większą od 2.
 
 \paragraph{}
 W teście Solovaya-Strassena użyto uogólnienia symbolu Legendre'a - symbolu Jacobiego.

 \subsection{Symbol Jacobiego} \label{sec:jacobi}
 Symbol Jacobiego jest uogólnieniem symbolu Legendre'a na liczby nieparzyste (niekoniecznie pierwsze).
 Jeśli rozkład liczby $n$ na czynniki pierwsze to:
 
 $p_1^{c_1}p_2^{c_2}\cdots p_k^{c_k}$,
 
 to symbol Jacobiego jest równy przez symbol Legendre'a:
 
 $\left( \frac a n \right) = \left( \frac a {p_1} \right)^{c_1} \left( \frac a {p_2} \right)^{c_2} \cdots \left( \frac a {p_k} \right)^{c_k}$.

 Można zauważyć, że jeśli $n$ jest pierwsze, symbol Jacobiego jest równy symbolowi Legendre'a.
 
 
 \subsection{Algorytm}
 Sprawdzenie, czy dana liczba naturalna $n$ jest pierwsza, odbywa się poprzez sprawdzenie, czy dla różnych wartości liczby naturalnej $a$ ($1 < a < n$) spełniona jest kongruencja:
 
 $\left(\frac{a}{n}\right) \equiv a^{(n-1)/2} \pmod n$.
 
 Jeśli dla którejkolwiek wartości $a$ powyższa kongruencja nie jest spełniona, to liczba $p$ jest liczbą złożoną - wartość symbolu Jacobiego dla pary liczb $(n, a)$  nie jest równa wartości symbolu Legendre'a dla tej pary (patrz Sekcja~\ref{sec:jacobi}).
 
 Kroki algorytmu są następujące:
 
 \begin{enumerate}
  \item Pobierz parametry wejściowe: $n$ i $k$.
  \item Powtórz $k$ razy:
   \begin{enumerate}
    \item Wybierz losowo $a$ ($1 < a < n$);
    \item Oblicz $x \leftarrow \left(\frac{a}{n}\right)$;
    \item Jeśli $x = 0$ lub $x \ne a^{(n-1)/2} \pmod n$, zwróć: \verb+złożona+.
   \end{enumerate}
  \item Zwróć: \verb+prawdopodobnie pierwsza+.
 \end{enumerate}
 
 Prawdopodobieństwo zwrócenia błędnego wyniku (tzn. zwrócenia \verb+prawdopodobnie pierwsza+ dla liczby złożonej) wynosi $2^{-k}$.


\section{Implementacja}
Aplikację implementującą test Solovaya-Strassena napisano w języku C\# (platforma .NET).
Kod źródłowy programu wraz komentarzami zawiera Załącznik~\ref{apx:source}.

Aplikacja została podzielona na następujące moduły:
\begin{itemize}
 \item \verb+SolovayStrassen.Logic+ - implementuje algorytm. Zawiera klasy: \verb+SolovayStrassenAlgorithm+ i \verb+JacobiAlgorithm+.
 \item \verb+SolovayStrassen.Logic.Tests+ - zawiera testy jednostkowe algorytmu (klasy: \verb+PrimeNumberTests+ i \verb+ComplexNumberTests+).
 \item \verb+SolovayStrassen.Console+ - punkt wejścia programu. Pobiera parametry wiersza poleceń ($n$ i $k$) i wywołuje algorytm.
\end{itemize}

Załącznik~\ref{apx:bin} zawiera pliki binarne programu i przykładowy skrypt uruchamiający.
Przykładowa komenda uruchamiająca program:

\begin{verbatim}
SolovayStrassen 274876858367 1000
\end{verbatim}


\section{Testowanie}
Aplikację przetestowano przy pomocy automatycznych testów jednostkowych.
Duże liczby pierwsze zaczerpnięto z: \emph{http://en.wikipedia.org/wiki/List\_of\_prime\_numbers}.

Wszystkie testy potwierdzają poprawność działania algorytmu.


\section*{Załączniki}
\appendix
\titleformat*{\section}{\normalfont}

\section{Katalog \emph{source/} - kod źródłowy aplikacji.} \label{apx:source}
\section{Katalog \emph{app/} - pliki binarne aplikacji wraz z przykładowym skryptem uruchamiającym.} \label{apx:bin}

\titleformat*{\section}{\Large\bfseries}


\end{document}
