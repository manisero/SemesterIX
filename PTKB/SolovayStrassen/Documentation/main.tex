\documentclass[a4paper,10pt]{article}
\usepackage[utf8]{inputenc}
\usepackage[MeX]{polski}
\usepackage{tikz}
\usepackage{url}
\usepackage{fullpage}
\usepackage{titlesec}
\usepackage{amsmath}

\title{Test pierwszości Solovaya-Strassena. \\ \small{Projekt z przedmiotu PTKB.}}
\author{Michał Aniserowicz, Jakub Turek}
\date{}

\begin{document}

\maketitle

\section{Opis zadania} \label{sec:task_desc}
Celem projektu jest zaimplementowanie probabilistycznego testu pierwszości Solovaya-Strassena.

\section{Teoria}
Probabilistyczny test pierwszości Solovaya-Strassena został opracowany przez Roberta M. Solovaya i Volkera Strassena.
Określa on, czy dana liczba jest liczbą złożoną czy prawdopodobnie pierwszą.

Podstawową wykorzystywaną przez niego własnością jest wykazany przez Eulera fakt, że dla każdej liczby pierwszej $p$ i dowolnej liczby naturalnej $a$, zachodzi:

$a^{(p-1)/2} \equiv \left(\frac{a}{p}\right) \pmod p$,

gdzie $\left(\frac{a}{p}\right)$ jest symbolem Legendre'a.

 \subsection{Symbol Legendre'a}
 Symbol Legendre'a to funkcja $\left( \frac a p \right)$ zdefiniowana następująco:

 $
 \left( \frac a p \right) = 
 \begin{cases}
   0 & \mbox{jeśli $a$ jest wielokrotnością $p$} \\
   1 & \mbox{jeśli istnieje takie $b$, że $b^2=a \mod p$} \\
  -1 & \mbox{jeśli nie istnieje żadne $b$ takie że $b^2=a \mod p$}
 \end{cases}
 $,

 gdzie $p$ jest liczbą pierwszą większą od 2.
 
 \paragraph{}
 W teście Solovaya-Strassena użyto uogólnienia symbolu Legendre'a - symbolu Jacobiego.

 \subsection{Symbol Jacobiego}
 Symbol Jacobiego jest uogólnieniem symbolu Legendre'a na liczby nieparzyste (niekoniecznie pierwsze).
 Jeśli rozkład liczby $n$ na czynniki pierwsze to:
 
 $p_1^{c_1}p_2^{c_2}\cdots p_k^{c_k}$,
 
 to symbol Jacobiego jest równy przez symbol Legendre'a:
 
 $\left( \frac a n \right) = \left( \frac a {p_1} \right)^{c_1} \left( \frac a {p_2} \right)^{c_2} \cdots \left( \frac a {p_k} \right)^{c_k}$.

 Można zauważyć, że jeśli $n$ jest pierwsze, symbol Jacobiego jest równy symbolowi Legendre'a.
 
 
 \subsection{Kroki algorytmu}
 


\section{Implementacja}
\begin{itemize}
 \item Język programowania: c\#, platforma  .NET.
\end{itemize}



\section{Testowanie}



\end{document}
