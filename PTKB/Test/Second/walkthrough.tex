\documentclass[a4paper,10pt, twocolumn]{article}
\usepackage[MeX]{polski}
\usepackage[utf8]{inputenc}
\usepackage{fullpage}
\usepackage{amsmath}

\title{[PTKB] Kolokwium 2 - opracowanie}
\author{}
\date{}

\begin{document}

\maketitle

\section{Zadanie 1.}

\textbf{Treść}: Ile razy trzeba wykonać protokół uwierzytelniania Fiata-Shamira by prawdopodobieństwo oszustwa było mniejsze od $10^{-1000}$?

\textbf{Rozwiązanie}: Prawdopodobieństwo udanego oszustwa po wykonaniu $n$ eksperymentów wynosi $(\frac{1}{2})^{n}$. Rozwiązujemy równanie $(\frac{1}{2})^{x} = 10^{-1000}$.

\begin{equation}
	\begin{array}{lcl} (\frac{1}{2})^{x} & = & 10^{-1000} \\ 2^{x} & = & 10^{1000} \\ x & = & \log_{2} 10^{1000} \\ x & = & 1000 \log_{2} 10 \\ x & \simeq & 3321.928 \end{array}
\end{equation}

Wybieramy $\lceil x \rceil = 3322$.

\section{Zadanie 7.}

\textbf{Treść}: Wykazać, że wielomian $x^{2} + 1$ jest nierozkładalny w~pierścieniu wielomianów $GF(3)[x]$, a~jest rozkładalny w~pierścieniu wielomianów $GF(2)[x]$.

\textbf{Rozwiązanie}: Wielomian drugiego stopnia można rozłożyć za pomocą dwóch wielomianów pierwszego stopnia, więc:

\begin{equation}
	\begin{array}{lcl} x^{2} + 1 & = & (ax + b) * (cx + d) \\ x^{2} + 1 & = & (ac)x^{2} + (ad + bc)x + bd \\ \end{array}
\end{equation}

Dla ciała $GF(3)[x]$, $b, d \in \left\{0, 1, 2\right\}$ oraz $a, c \in \left\{1, 2\right\}$ (bo wielomian musi być rozkładalny). Rozważmy wszystkie możliwe wartości $(ad + bc) \bmod{3}$. Jeżeli $(ad + bc) \equiv 0 \mod 3 \Rightarrow a = 0 \wedge c = 0$, co jest sprzeczne z~dziedziną, a~więc wielomian nie może być rozkładalny.

Dla ciała $GF(2)[x]$, $b, d \in \left\{0, 1\right\}$ oraz $a, c \in \left\{1\right\}$. Jeżeli $(b + d) \equiv 0 \mod 2 \Rightarrow (b = 0 \wedge d = 0) \vee (b = 1 \wedge d = 1)$. Dla drugiego przypadku otrzymujemy w~$GF(2)[x]$:

\begin{equation}
	x^{2} + 1 \equiv (x+1) * (x+1)
\end{equation}

Zatem wielomian jest rozkładalny.

\end{document}
