\documentclass[a4paper,10pt, twocolumn]{article}
\usepackage[MeX]{polski}
\usepackage[utf8]{inputenc}
\usepackage{fullpage}
\usepackage{amsmath}

\title{[PTKB] Kolokwium 2 - opracowanie}
\author{}
\date{}

\begin{document}

\maketitle

\section{Zadanie 1.}

\textbf{Treść}: Ile razy trzeba wykonać protokół uwierzytelniania Fiata-Shamira by prawdopodobieństwo oszustwa było mniejsze od $10^{-1000}$?

\textbf{Rozwiązanie}: Prawdopodobieństwo udanego oszustwa po wykonaniu $n$ eksperymentów wynosi $(\frac{1}{2})^{n}$. Rozwiązujemy równanie $(\frac{1}{2})^{x} = 10^{-1000}$.

\begin{equation}
	\begin{array}{lcl} (\frac{1}{2})^{x} & = & 10^{-1000} \\ 2^{x} & = & 10^{1000} \\ x & = & \log_{2} 10^{1000} \\ x & = & 1000 \log_{2} 10 \\ x & \simeq & 3321.928 \end{array}
\end{equation}

Wybieramy $\lceil x \rceil = 3322$.

\end{document}
