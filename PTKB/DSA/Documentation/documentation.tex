\documentclass{article}
\usepackage[MeX]{polski}
\usepackage[utf8]{inputenc}
\usepackage{fullpage}
\usepackage{hyperref}
\usepackage{amsmath}
\usepackage{listings}

\title{Dokumentacja końcowa projektu \\ \Large{Algorytm DSA}}
\author{Jakub Turek}
\date{}

\begin{document}

    \maketitle

    \section*{Podstawy teoretyczne}
    
        FIPS\footnote{Skrót od \textbf{F}ederal \textbf{I}nformation \textbf{P}rocessing \textbf{S}tandard (ang. federalny standard przetwarzania informacji).} jest zbiorem, w~którym opisane są publiczne standardy bezpieczeństwa używane przez federalny rząd Stanów Zjednoczonych. Oficjalnym standardem podpisywania wiadomości cyfrowych zamieszczonym w~FIPS jest DSS (ang. Digital Signature Standard). DSS opiera się o~algorytm DSA (ang. Digital Signature Algorithm).
    
        Standard DSS (wraz z~algorytmem DSA) został opisany w~dokumencie FIPS PUB 186\footnote{\url{http://www.itl.nist.gov/fipspubs/fip186.htm}.}. Na potrzeby projektu zaimplementowany został oryginalny algorytm opublikowany w~1994 roku, który wykorzystuje funkcję skrótu SHA.
        
    \subsection*{Algorytm}
    
        \paragraph*{Generacja kluczy} Należy wybrać parametry:
                    
        \begin{itemize}
            \item Liczba pierwsza $p$ w~pierścieniu reszt modulo $a$, gdzie $2^{L-1} < p < 2^{L}$ oraz $512 \leq L \leq 1024$ i~$L$ jest wielokrotnością 64.
            \item Liczba pierwsza $q$ będąca dzielnikiem liczby $p - 1$ w~pierścieniu reszt modulo $a$, gdzie $2^{159} < q < 2^{160}$.
            \item Liczba $g = h^{\frac{p-1}{q}} \pmod p$, gdzie $h$ jest dowolną liczbą naturalną spełniającą warunek $1 < h < p - 1$ taką, że $h^{\frac{p-1}{q}} \pmod p > 1$ (czyli $g$ ma rząd $q \pmod p$).
            \item Losowo wygenerowana liczba $x$ z~przedziału $0 < x < q$.
            \item Liczba $y = g^{x} \pmod p$.
            \item Losowo wygenerowana liczba $k$ z~przedziału $0 < k < q$.
        \end{itemize}
            
        Liczby $p$, $q$ oraz $g$ są publiczne. Klucz prywatny użytkownika to $x$, natomiast klucz publiczny użytkownika to $y$. Parametr $k$ musi być obliczany dla każdego nowego podpisu. Klucze są wielokrotnego użytku.
        
        \paragraph*{Podpisywanie wiadomości} Podpisem wiadomości $M$ jest para liczb $(r, s)$ obliczanych według poniższego wzoru:
        
        \begin{equation*}
            \begin{array}{lcl}
                r & = & (g^{k} \mod p) \mod q \\
                s & = & (k^{-1} (SHA(M) + xr)) \mod q \\
            \end{array}
        \end{equation*}
        
        \noindent gdzie $k^{-1}$ jest odwrotnością liczby $k$ w~pierścieniu reszt modulo $q$ (czyt. $k \cdot k^{-1} \equiv 1 \pmod q$). 
        
        Opcjonalnie można zweryfikować, czy $r \neq 0$ i~$s \neq 0$. Jeżeli jeden z~warunków nie jest spełniony, należy wygenerować podpis od nowa. Sytuacja taka nie powinna się jednak zdarzyć dla prawidłowo wygenerowanych kluczy.
        
        \paragraph*{Weryfikacja podpisu} Zakładamy, że otrzymaliśmy zestaw $(M', (r', s'))$ składający się z~wiadomości oraz podpisu tej wiadomości. Aby zweryfikować podpis należy:
        
        \begin{enumerate}
            \item Dokonać sprawdzenia, że $0 < r' < q$, a~ponadto $0 < s' < q$.
            \item Obliczyć poniższe wartości:
            
                \begin{equation*}
                    \begin{array}{lcl}
                        w & = & (s')^{-1} \pmod q \\
                        u_{1} & = & SHA(M') \cdot w \pmod q \\
                        u_{2} & = & r' \cdot w \pmod q \\
                        v & = & (g^{u_{1}} \cdot y^{u_{2}} \mod p) \pmod q \\
                    \end{array}
                \end{equation*}
            \item Sprawdzić, czy $v = r'$.
            \item Podpis jest prawidłowy, jeżeli warunek z~punktu 3. jest spełniony.
        \end{enumerate}
        
    \subsection*{Dowód poprawności}
    
        \paragraph{Lemat} 
            
            Niech $p$ i~$q$ będą liczbami pierwszymi takimi, że $q$ dzieli $p - 1$, $h$ jest dodatnią liczbą całkowitą mniejszą niż $p$ i~spełnione jest równanie $g = h^{\frac{p-1}{q}} \mod p$. Wtedy $g^{q} \mod p = 1$ i, jeżeli $m \mod q = n \mod q$, wtedy $g^{m} \mod p = g^{n} \mod p$.
            
        \paragraph{Dowód} 
        
            Z~Małego Twierdzenia Fermata wynika równość: 
            
            \begin{equation*}
                g^{q} \mod p = (h^{\frac{p - 1}{q}} \mod p) q \mod p = h^{p-1} \mod p = 1                            
            \end{equation*}

            Jeśli $m \mod q = n \mod q$, przykładowo $m = n + kq$ dla pewnej liczby całkowitej $k$. Wtedy:

            \begin{equation*}
                g^{m} \mod p = g^{n+kq} \mod p = (g^{n}g^{kq}) \mod p = ((g^{n} \mod p)(g^{q} \mod p)^{k}) \mod p = g^{n} \mod p
            \end{equation*}

            \noindent ponieważ $g^{q} \mod p = 1$.
            
        \paragraph{Twierdzenie}
        
            Jeżeli $M' = M$, $r' = r$ oraz $s' = s$ w~podpisie DSA, wtedy $v = r'$.
            
        \paragraph{Dowód}
        
            Mamy:
            
            \begin{equation*}
                \begin{array}{rcl}
                    w = & (s')^{-1} \mod q & = s^{-1} \mod q \\ 
                    u_{1} = & (SHA(M') \cdot w) \mod q & = (SHA(M) \cdot w) \mod q \\
                    u_{2} = & (r' \cdot w) \mod q & = r \cdot w \mod q
                \end{array}
            \end{equation*}
            
            \noindent $y = g^{x} \mod p$, zatem wykorzystując lemat:
            
            \begin{equation*}
                \begin{array}{rl}
                    v = & (g^{u_{1}} y^{u_{2}} \mod p) \mod q = \\
                    = & (g^{SHA(M) \cdot w} y^{r \cdot w} \mod p) \mod q = \\
                    = & (g^{SHA(M) \cdot w} g^{xrw} \mod p) \mod q = \\
                    = & (g^{(SHA(M) + xr) \cdot w} \mod p) \mod q\\
                \end{array}
            \end{equation*}
            
            Ponadto wiemy, że $s = (k^{-1} (SHA(M) + xr)) \mod q$. Stąd wynika, że:
            
            \begin{equation*}
                \begin{array}{c}
                    w = (k(SHA(M) + xr)^{-1}) \mod q \\
                    ((SHA(M) + xr) \cdot w) \mod q = k \mod q
                \end{array}
            \end{equation*}      
            
            Zatem, powołując się na lemat, możemy stwierdzić, że:
            
            \begin{equation*}
                v = (g^{k} \mod p) \mod q = r = r'
            \end{equation*}
            
            \noindent co należało udowodnić.

    \section*{Implementacja}
    
        Implementacja została oparta o~sugestie zawarte w~załącznikach do dokumentu FIPS PUB 186. Przedstawiony został w~nich algorytm generacji kluczy. Dodatkowo, do implementacji wykorzystane zostały komponenty biblioteki \verb+PyCrypto+\footnote{\url{https://www.dlitz.net/software/pycrypto/}.}. Dzięki temu możliwe było uniknięcie implementacji funkcjonalności, które są przedmiotami innych projektów, takich jak funkcja skrótu \verb+SHA+ czy test pierwszości Millera-Rabina.
        
        \paragraph{Generacja liczb p~i~q} Załącznik drugi (punkt 2.2) do dokumentu FIPS PUB 186 sugeruje, aby w pierwszej kolejności wygenerować liczbę $q$, w~oparciu o~dowolną sekwencję długości przynajmniej 160 bitów nazwaną ziarnem. 
        
        \noindent\begin{table}[ht!]
            \begin{tabular}{lr}
                \begin{minipage}[t]{0.45\textwidth}
                    \begin{verbatim}
seed = os.urandom(20)


hash1 = SHA.new(seed).digest()
hash2 = SHA.new(long_to_bytes(
    bytes_to_long(seed)+1)).digest()
    
q = 0L

for i in range(0, 20):
    c = ord(hash1[i]) ^ ord(hash2[i])

    if i == 0:
        c |= 128

    if i == 19:
        c |= 1

    q = q * 256 + c


while not isPrime(q):
    q += 2

if pow(2, 159L) < q < pow(2, 160L):
    return seed, q
                    \end{verbatim}
                \end{minipage}
                
                &
        
                \begin{minipage}[t]{0.45\textwidth}
                    \noindent Wybieramy losowy ciąg znaków o długości 20 bajtów (160 bitów). \\
                    
                    \noindent Obliczamy funkcję skrótu dla ziarna...
                    
                    \noindent ... oraz ziarna powiększonego o~1. \\ \\ \\ \\
                                        
                    \noindent Dla każdego bajtu ziarna...
                    
                    \noindent ... obliczamy wartość funkcji XOR. \\
                    
                    \noindent Jeżeli najbardziej znaczący bajt...
                    
                    \noindent ... to ustawiamy jeden na pierwszym bicie. \\
                    
                    \noindent Jeżeli najmniej znaczący bajt...
                    
                    \noindent ... to ustawiamy jeden na ostatnim bicie. \\
                    
                    \noindent Przesuwamy obliczoną liczbę o~bajt w~lewo i~dołączamy z~prawej obliczony w~pętli bajt. \\
                    
                    \noindent Dodajemy kolejne dwójki (nie zmienia się ostatni bit) aż do uzyskania liczby pierwszej. \\
                    
                    \noindent Jeżeli uzyskane $q$ zawiera się w~narzuconym przedziale to zwracamy ziarno oraz $q$.
                    
                \end{minipage}
            
                \\
            
            \end{tabular}
        
        \end{table}
        
        W~stosunku do części algorytmu generacji $q$ przedstawionej w~krokach 1. - 5. zastosowano jedną optymalizację. W~przypadku, gdy obliczone w~operacji XOR $q$ nie jest pierwsze, wtedy poszukiwana jest kolejna (większa od znalezionej) liczba pierwsza. W~oryginalnym dokumencie algorytm byłby powtarzany od nowa.
        
        Kolejnym krokiem jest generacja liczby $p$. Liczba $p$ dla danej liczby $q$ może zostać wygenerowana przy pomocy kroków 6. - 15. algorytmu przedstawionego w~sekcji 2.2 załącznika drugiego do dokumentu FIPS PUB 186.
        
        \pagebreak

        \noindent\begin{table}[ht!]
            \begin{tabular}{lr}
                \begin{minipage}[t]{0.45\textwidth}
                    \begin{verbatim}
while True:
    seed, q = generate_q()
    n = divmod(bits - 1, 160)[0]
    counter, offset, v = 0, 2, {}
    b = q >> 5 & 15
    twopowbitsone = pow(2L, bits - 1)

    while counter < 4096:
        for k in range(0, n+1):
            v[k] = bytes_to_long(SHA.new(
                seed + str(offset) + 
                str(k)).digest())



            w = v[n] % pow(2L, b)

            for k in range(n - 1, -1, -1):
                w = (w << 160L) + v[k]

            x = w + twopowbitsone
            
            c = x % (2 * self.q)
            
            p = x - (c - 1)

            if twopowbitsone <= p and 
                isPrime(p):
                break

            counter += 1
            
            offset += n + 1

    if counter < 4069:
        break
                    \end{verbatim}
                \end{minipage}
                
                &
        
                \begin{minipage}[t]{0.45\textwidth}
                    \noindent Powtarzamy aż do generacji poprawnego $p$. \\
                    
                    \noindent Obliczamy $n$ ze wzoru $L - 1 = n * 160 + b$. \\
                    
                    \noindent Obliczamy $b$. \\ \\
                                        
                    \noindent Dla danego $q$ generujemy $p$ ($2^{12}$ prób). \\
                    
                    \noindent Obliczamy $V[k]$ ze wzoru $V[k] = [SHA(SEED + offset + k) \mod 2^{g}]$. \\
                    
                    \noindent Obliczamy W ze wzoru $V_{0} + V_{1} * 2^{160} + \cdots + (V_{n} \mod 2^{b}) \cdot 2^{n \cdot 160}$... \\
                    
                    \noindent ... najpierw obliczając skrajnie prawy wyraz... \\
                    
                    \noindent ... a~następnie obliczając kolejne wyrazy idąc w~lewą stronę. \\
                    
                    \noindent Obliczamy $X = W + 2^{L-1}$. \\
                    
                    \noindent Obliczamy $c = X \mod 2 \cdot q$. \\
                    
                    \noindent Obliczamy $p = X - (c - 1)$. \\
                    
                    \noindent Jeśli uzyskana liczba jest pierwsza to poprawnie wygenerowaliśmy $p$ i~przerywamy algorytm. \\ \\
                    
                    \noindent Inkrementujemy licznik. \\
                    
                    \noindent Zwiększamy przesunięcie.  \\
                    
                    \noindent Jeżeli liczba $p$ jest poprawna dla danego $q$ to kontynuujemy generację klucza.
                    
                \end{minipage}
            
                \\
            
            \end{tabular}
        \end{table}
        
        \paragraph{Generacja klucza} Po wygenerowaniu liczb $p$ oraz $q$ przystępujemy do generacji $(g, x, y)$. 
        
        \noindent\begin{table}[ht!]
            \begin{tabular}{lr}
                \begin{minipage}[t]{0.45\textwidth}
                    \begin{verbatim}
while True:
    h = bytes_to_long(os.urandom(bits)) 
        % (p - 1)
    g = pow(h, divmod(p - 1, q)[0], p)

    if 1 < h < p - 1 and g > 1:
        break
                    \end{verbatim}
                \end{minipage}
                
                &
        
                \begin{minipage}[t]{0.45\textwidth}                    
                    \noindent \\ Generujemy losowe $h$ z~pierścienia modulo $p - 1$. \\
                    
                    \noindent Obliczamy $g$ jako $h$ podniesione do części całkowitej z~dzielenia $\frac{p-1}{q}$ w~pierścieniu modulo $p$. \\
                    
                    \noindent Sprawdzamy czy $g$~i~$h$ spełniają zadane warunki.
                \end{minipage}
            
                \\
            
            \end{tabular}
        \end{table}
            
        \noindent\begin{table}[ht!]
            \begin{tabular}{lr}
                \begin{minipage}[t]{0.45\textwidth}
                    \begin{verbatim}
while True:
    x = bytes_to_long(os.urandom(20))

    if 0 < x < q:
        break

y = pow(g, x, p)
                    \end{verbatim}
                \end{minipage}
                
                &
        
                \begin{minipage}[t]{0.45\textwidth}                    
                    \noindent \\ Generujemy losową liczbę $x$... \\
                    
                    \noindent ... która spełnia warunek $0 < x < q$. \\ \\
                    
                    \noindent Obliczamy klucz publiczny $y$.
                \end{minipage}
            
                \\
            
            \end{tabular}
        \end{table}
        
        \paragraph{Podpisywanie wiadomości} Podpisywanie wiadomości zostało zaimplementowane zgodnie z~założeniami algorytmu DSA.
        
        \noindent\begin{table}[ht!]
            \begin{tabular}{lr}
                \begin{minipage}[t]{0.45\textwidth}
                    \begin{verbatim}
def sign(self, message):
    m = bytes_to_long(SHA.new(message)
        .digest())
    k = randint(1, q - 1)
    
    inverse_k = inverse(k, q)
    
    r = pow(g, k, p) % q
    
    s = (inverse_k * (m + x * r)) % q

    return r, s
                    \end{verbatim}
                \end{minipage}
                
                &
        
                \begin{minipage}[t]{0.45\textwidth}                    
                    \noindent \\ Obliczamy wartość skrótu wiadomości \verb+message+. \\
                    
                    \noindent Wybieramy losowe $k$ z~przedziału $(0, q)$. \\
                    
                    \noindent Obliczamy odwrotność $k$ w~pierścieniu modulo $q$. \\
                    
                    \noindent Obliczamy $r = (g^{k} \mod p) \mod q$. \\
                    
                    \noindent Obliczamy $s = (k^{-1} (SHA(M) + xr)) \mod q$.
                \end{minipage}
            
                \\
            
            \end{tabular}
        \end{table}
        
        \paragraph{Weryfikacja podpisu} Podobnie, weryfikacja podpisu również została zaimplementowana zgodnie z~założeniami algorytmu DSA.
        
        \noindent\begin{table}[ht!]
            \begin{tabular}{lr}
                \begin{minipage}[t]{0.45\textwidth}
                    \begin{verbatim}
def verify(self, message, r, s):
    m = bytes_to_long(SHA.new(message)
        .digest())

    if not (0 < r < q) or not (0 < s < q):
        return False

    w = inverse(s, q)
    
    u1 = (m * w) % q
    u2 = (r * w) % q
    
    v = ((pow(g, u1, p) * pow(y, u2, p)) 
        % p) % q

    return v == r
                    \end{verbatim}
                \end{minipage}
                
                &
        
                \begin{minipage}[t]{0.45\textwidth}                    
                    \noindent \\ Obliczamy skrót wiadomości $message$. \\ \\
                    
                    \noindent Jeżeli $r$ lub $s$ nie znajdują się w~zadanych przedziałach, to podpis jest nieprawidłowy. \\
                    
                    \noindent Obliczamy $w = s^{-1} \mod q$. \\
                    
                    \noindent Obliczamy $u_{1} oraz u_{2}$. \\ \\
                    
                    \noindent Obliczamy $v = (g^{u_{1}} \cdot y^{u_{2}} \mod p) \mod q$. \\ \\
                    
                    \noindent Dokonujemy weryfikacji.
                \end{minipage}
            
                \\
            
            \end{tabular}
        \end{table}
        
    \section*{Testy}
    
        Scenariusz testów opisany jest testami jednostkowymi klasy \verb+DSAKey+. Znajdują się one w~module \verb+DSAKeyTests+.
        
        Podstawowym testem na poprawne działanie algorytmu jest podpisanie wiadomości kluczem prywatnym z~wykorzystaniem algorytmu DSA, a~następnie zweryfikowanie podpisu tej samej wiadomości przy użyciu klucza publicznego. Opisuje to scenariusz \verb+test_positive_workflow()+. W~scenariuszu tym:
        
        \begin{enumerate}
            \item Generowana jest losowa para kluczy prywatny oraz publiczny.
            \item Przy pomocy klucza prywatnego podpisywana jest wiadomość \verb+Test message+.
            \item Przy pomocy klucza publicznego weryfikowany jest podpis wiadomości \verb+Test message+ uzyskany w~punkcie 2.
            \item Do poprawnego zakończenia testu wymagane jest, aby podpis był prawidłowy.
        \end{enumerate}

        W~dalszej kolejności należy upewnić się, że jakakolwiek modyfikacja wiadomości powoduje, że podpis przestaje być ważny. Sprawdzenie takie wykonywane jest w~trzech osobnych scenariuszach:
        
        \begin{itemize}
            \item Zmodyfikowana wiadomość posiada zmienioną literę na jednej pozycji.
            \item Zmodyfikowana wiadomość ma zamienioną kolejność dwóch liter.
            \item Zmodyfikowana wiadomość została skrócona.
        \end{itemize}
        
\noindent Scenariusze te mają za zadanie sprawdzić, że każda drobna modyfikacja wiadomości powoduje, iż podpis traci ważność.

        Pierwszy ze scenariuszy negatywnych (zmieniona litera na jednej pozycji) przebiega następująco:
        
        \begin{enumerate}
            \item Generowana jest losowa para kluczy prywatny oraz publiczny.
            \item Przy pomocy klucza prywatnego podpisywana jest wiadomość \verb+Test message+.
            \item Przy pomocy klucza publicznego weryfikowany jest podpis wiadomości uzyskany w~punkcie 2., przy czym wiadomość $M$ zostaje zastąpiona treścią \verb+Best message+.
            \item Do poprawnego zakończenia testu wymagane jest, aby podpis został odrzucony.
        \end{enumerate}

        Kolejny ze scenariuszy negatywnych (zamieniona kolejność dwóch liter) przebiega następująco:
        
        \begin{enumerate}
            \item Generowana jest losowa para kluczy prywatny oraz publiczny.
            \item Przy pomocy klucza prywatnego podpisywana jest wiadomość \verb+Test message+.
            \item Przy pomocy klucza publicznego weryfikowany jest podpis wiadomości uzyskany w~punkcie 2., przy czym wiadomość $M$ zostaje zastąpiona treścią \verb+Tset message+.
            \item Do poprawnego zakończenia testu wymagane jest, aby podpis został odrzucony.
        \end{enumerate}        
        
        Ostatni ze scenariuszy negatywnych (skrócona wiadomość) przebiega następująco:
        
        \begin{enumerate}
            \item Generowana jest losowa para kluczy prywatny oraz publiczny.
            \item Przy pomocy klucza prywatnego podpisywana jest wiadomość \verb+Test message+.
            \item Przy pomocy klucza publicznego weryfikowany jest podpis wiadomości uzyskany w~punkcie 2., przy czym wiadomość $M$ zostaje zastąpiona treścią \verb+message+.
            \item Do poprawnego zakończenia testu wymagane jest, aby podpis został odrzucony.
        \end{enumerate}
        
        Dodatkowo należy upewnić się, że zmiana podpisu wiadomości (a~więc jednej lub obu liczb z~pary $(r, s)$) również spowoduje, że podpis zostanie odrzucony. Testowane są trzy przypadki:
        
        \begin{itemize}
            \item Wybierana jest losowa wartość $r'$.
            \item Wybierana jest losowa wartość $s'$.
            \item Wybierana jest losowa wartość pary $(r', s')$.
        \end{itemize}

        \noindent Aby ,,uprawdopodobnić'', że losowa wartość klucza (lub jego części) zostanie uznana za poprawną w~każdym teście generowane są zmienione liczby o~długości bitowej zgodnej z~oryginalnymi długościami bitowymi $(r, s)$.

        Pierwszy ze scenariuszy negatywnych dla zmienionego podpisu $(r, s)$ wiadomości przebiega następująco:
        
        \begin{enumerate}
            \item Generowana jest losowa para kluczy prywatny oraz publiczny.
            \item Przy pomocy klucza prywatnego podpisywana jest wiadomość \verb+Test message+.
            \item Generowana jest losowa wartość $r'$ o~długości bitowej równej długości bitowej $r$.
            \item Przy pomocy klucza publicznego weryfikowany jest podpis wiadomości \verb+Test message+ o~wartości $(r', s)$.
            \item Test zakończy się poprawnie wtedy i~tylko wtedy, gdy weryfikacja zakończy się tym samym wynikiem co porównanie $r' == r$.
        \end{enumerate}
        
        Kolejny ze scenariuszy negatywnych dla zmienionego podpisu $(r, s)$ wiadomości przebiega następująco:
        
        \begin{enumerate}
            \item Generowana jest losowa para kluczy prywatny oraz publiczny.
            \item Przy pomocy klucza prywatnego podpisywana jest wiadomość \verb+Test message+.
            \item Generowana jest losowa wartość $s'$ o~długości bitowej równej długości bitowej $s$.
            \item Przy pomocy klucza publicznego weryfikowany jest podpis wiadomości \verb+Test message+ o~wartości $(r, s')$.
            \item Test zakończy się poprawnie wtedy i~tylko wtedy, gdy weryfikacja zakończy się tym samym wynikiem co porównanie $s' == s$.
        \end{enumerate}
        
        Ostatni ze scenariuszy negatywnych dla zmienionego podpisu $(r, s)$ wiadomości przebiega następująco:
        
        \begin{enumerate}
            \item Generowana jest losowa para kluczy prywatny oraz publiczny.
            \item Przy pomocy klucza prywatnego podpisywana jest wiadomość \verb+Test message+.
            \item Generowana jest losowa wartość $r'$ o~długości bitowej równej długości bitowej $r$.
            \item Generowana jest losowa wartość $s'$ o~długości bitowej równej długości bitowej $s$.
            \item Przy pomocy klucza publicznego weryfikowany jest podpis wiadomości \verb+Test message+ o~wartości $(r', s')$.
            \item Test zakończy się poprawnie wtedy i~tylko wtedy, gdy weryfikacja zakończy się tym samym wynikiem co porównanie $r' == r \wedge s' == s$.
        \end{enumerate}
        
\end{document}