\documentclass{article}
\usepackage[MeX]{polski}
\usepackage[utf8]{inputenc}
\usepackage{fullpage}
\usepackage{hyperref}

\title{Dokumentacja końcowa projektu \\ \Large{Algorytm DSA}}
\author{Jakub Turek}
\date{}

\begin{document}

    \maketitle

    \section*{Podstawy teoretyczne}
    
        FIPS\footnote{Skrót od \textbf{F}ederal \textbf{I}nformation \textbf{P}rocessing \textbf{S}tandard (ang. federalny standard przetwarzania informacji).} jest zbiorem, w~którym opisane są publiczne standardy bezpieczeństwa używane przez federalny rząd Stanów Zjednoczonych. Oficjalnym standardem podpisywania wiadomości cyfrowych zamieszczonym w~FIPS jest DSS (ang. Digital Signature Standard). DSS opiera się o~algorytm DSA (ang. Digital Signature Algorithm).
    
        Standard DSS (wraz z~algorytmem DSA) został opisany w~dokumencie FIPS PUB 186\footnote{\url{http://www.itl.nist.gov/fipspubs/fip186.htm}.}. Na potrzeby projektu zaimplementowany został oryginalny algorytm opublikowany w~1994 roku, który wykorzystuje funkcję skrótu SHA.
        
\end{document}