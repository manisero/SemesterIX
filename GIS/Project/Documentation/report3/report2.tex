\documentclass[a4paper,10pt]{article}
\usepackage[utf8]{inputenc}
\usepackage[MeX]{polski}
\usepackage{tikz}
\usepackage{url}
\usepackage{fullpage}
\usepackage{dirtree}

\title{Grafy i Sieci. Sprawozdanie 3. \\ \small{SK11 Kolorowanie grafu za pomocą przeszukiwania z tabu.}}
\author{Michał Aniserowicz, Jakub Turek}
\date{}

\begin{document}

\maketitle

\section*{Temat projektu}

SK11 Kolorowanie grafu za pomocą przeszukiwania z tabu.

\section*{Dokumentacja kodu źródłowego}

Kod źródłowy projektu został stworzony w~języku Python. Program jest kompatybilny z~wersją \verb+2.7.x+ interpretera. Aplikacja testowana była w~Pythonie w~wersji \verb+2.7.5+, pod kontrolą systemu \verb+OS X 10.9+ (\verb+Mavericks+). Do uruchomienia testów jednostkowych wymagane jest zainstalowanie biblioteki \verb+Mock+\footnote{Biblioteka została wcielona do specyfikacji języka począwszy od wersji 3.3.} w~wersji \verb+1.0.1+.

Ogólna struktura kodu źródłowego została przedstawiona na poniższym diagramie.

\dirtree{%
 .1 /.
 .2 aspiration\_criteria.
 .3 aspiration\_criteria.py.
 .2 evaluation.
 .3 cost\_evaluator.py. 
 .2 graph.
 .3 graph\_cloner.py.
 .3 node.py.
 .3 node\_iterator.py.
 .2 input.
 .3 dimacs\_input\_reader.py.
 .3 input\_reader.py.
 .3 input\_reader\_factory.py.
 .2 memory.
 .3 memory.py.
 .2 permutation.
 .3 color\_permutator.py.
 .3 fast\_color\_permutator.py.
 .2 progress.
 .3 progress\_writer.py.
 .2 search.
 .3 search\_performer.py.
 .2 stop\_criteria.
 .3 stop\_criteria.py.
 .2 test.
 .3 ....
 .2 validation.
 .3 coloring\_validator.py.
 .3 connection\_validator.py.
 .2 main.py.
}

\end{document}
