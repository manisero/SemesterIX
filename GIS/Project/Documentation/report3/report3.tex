\documentclass[a4paper,10pt]{article}
\usepackage[utf8]{inputenc}
\usepackage[MeX]{polski}
\usepackage{tikz}
\usepackage{url}
\usepackage{fullpage}
\usepackage{dirtree}

\title{Grafy i Sieci. Sprawozdanie 3. \\ \small{SK11 Kolorowanie grafu za pomocą przeszukiwania z tabu.}}
\author{Michał Aniserowicz, Jakub Turek}
\date{}

\begin{document}

\maketitle

\section*{Temat projektu}

SK11 Kolorowanie grafu za pomocą przeszukiwania z tabu.

\section*{Dokumentacja kodu źródłowego}

Kod źródłowy projektu został stworzony w~języku Python. Program jest kompatybilny z~wersją \verb+2.7.x+ interpretera. Aplikacja testowana była w~Pythonie w~wersji \verb+2.7.5+, pod kontrolą systemu \verb+OS X 10.9+ (\verb+Mavericks+). Do uruchomienia testów jednostkowych wymagane jest zainstalowanie biblioteki \verb+Mock+\footnote{Biblioteka została wcielona do specyfikacji języka począwszy od wersji 3.3.} w~wersji \verb+1.0.1+.

Ogólna struktura kodu źródłowego została przedstawiona na poniższym diagramie.

\dirtree{%
 .1 /.
 .2 aspiration\_criteria.
 .3 aspiration\_criteria.py.
 .2 evaluation.
 .3 cost\_evaluator.py. 
 .2 graph.
 .3 graph\_cloner.py.
 .3 node.py.
 .3 node\_iterator.py.
 .2 input.
 .3 dimacs\_input\_reader.py.
 .3 input\_reader.py.
 .3 input\_reader\_factory.py.
 .2 memory.
 .3 memory.py.
 .2 permutation.
 .3 color\_permutator.py.
 .3 fast\_color\_permutator.py.
 .2 progress.
 .3 progress\_writer.py.
 .2 search.
 .3 search\_performer.py.
 .2 stop\_criteria.
 .3 stop\_criteria.py.
 .2 test.
 .3 ....
 .2 validation.
 .3 coloring\_validator.py.
 .3 connection\_validator.py.
 .2 main.py.
}
 
\subsection*{Reprezentacja grafu}

Graf reprezentowany jest z~wykorzystaniem klasy \verb+Node+ reprezentującej wierzchołek. Ponieważ, z~założenia, aplikacja operuje wyłącznie na grafach spójnych nie ma znaczenia, od którego wierzchołka rozpoczynamy analizę struktury.

\noindent\begin{table}[ht!]
            \begin{tabular}{lr}
                \begin{minipage}[t]{0.55\textwidth}
                    \begin{verbatim}
class Node:
  Id = 0

  def __init__(self, color=None, 
    node_id=None, previous_color=None):
    
    self.edges = []
    self.color = color

    if node_id is not None:
      self.node_id = node_id
    else:
      self.node_id = Node.Id
      Node.Id += 1

    self.previous_color = self.color

    if previous_color is not None:
      self.previous_color = previous_color

  def add_edges(self, nodes):
    for node in nodes:
      if node not in self.edges:
        self.edges.append(node)

      if self not in node.edges:
        node.edges.append(self)

  def iterator(self):
    return NodeIterator(self)

  def get_node_of_id(self, node_id):
    for node in self.iterator():
      if node.node_id == node_id:
        return node

  def node_count(self):
    return sum(1 for _ in self.iterator())

  def get_colors_count(self):
    colors = set()

    for node in self.iterator():
      colors.add(node.color)

    return len(colors)
                    \end{verbatim}
                \end{minipage}
                
                &
        
                \begin{minipage}[t]{0.45\textwidth}
                    \noindent Metoda \verb+init+ służy do konstrukcji węzła. Węzeł posiada następujące składowe:
                    \begin{itemize}
                        \item \verb+edges+ lista wierzchołków połączonych z~danym węzłem,
                        \item \verb+color+ kolor wierzchołka,
                        \item \verb+node_id+ identyfikator wierzchołka,
                        \item \verb+previous_color+ poprzedni kolor wierzchołka używany do wyznaczania permutacji.
                    \end{itemize}
\\
                    
                    \noindent Identyfikator, jak również kolor wierzchołka, mogą być dowolnego typu (liczba, ciąg znaków...). Identyfikatory mogą, ale nie muszą być nadawane automatycznie - są wtedy typu liczbowego. Kolejne identyfikatory pobierane są ze zmiennej ,,statycznej'' \verb+Id+. \\ \\
                    
                    \noindent Metoda \verb+add_edges+ pozwala na łączenie wierzchołka z~innymi wierzchołkami. Implementacja została przygotowana dla grafów nieskierowanych, a~więc podczas dodawania krawędzi tworzone jest od razu wiązanie dwustronne. \\ \\ \\
                                        
                    \noindent Do poruszania się po grafie wykorzystywany jest iterator, który korzysta z~algorytmu DFS. \\
                    
                    \noindent Metoda \verb+get_node_of_id+ pozwala na dojście do dowolnego wierzchołka po identyfikatorze. \\ \\ \\
                    
                    \noindent Metoda \verb+node_count+ zlicza liczbę wierzchołków w~grafie. \\
                    
                    \noindent Metoda \verb+get_colors_count+ zwraca liczbę kolorów, którymi w~chwili obecnej pokolorowany jest graf.
                                        
                \end{minipage}
            
                \\
            
            \end{tabular}
        
        \end{table}
        
Klasa \verb+NodeIterator+ dostarcza interfejs iteratora dla wierzchołka grafu. Udostępnia ona metodę \verb+next+, która dla danego wierzchołka zwraca kolejny w~porządku przeszukiwania w~głąb. Przeszukiwanie w~głąb oznacza, że w~pierwszej kolejności przechodzimy do pierwszego dziecka danego wierzchołka, a~dopiero po powrocie algorytmu do tego samego wierzchołka przeglądamy jego kolejne dziecko. Wykorzystanie wzorca iteratora pozwala na przeglądanie grafu w~wygodny sposób - używając do tego pętli \verb+for+.

Oprócz narzędzia do przeglądania grafu zaimplementowana została też metoda do kopiowania całego grafu. Jest ona zawarta w~metodzie \verb+clone+ klasy \verb+GraphCloner+. Klonowanie grafu jest przydatne podczas wyznaczania możliwych permutacji kolorów. Wystarczy powielić cały graf i~zmienić barwę analizowanego wierzchołka.

\subsection*{Funkcja kosztu}

\noindent\begin{table}[ht!]
            \begin{tabular}{lr}
                \begin{minipage}[t]{0.55\textwidth}
                    \begin{verbatim}
class CostEvaluator:
  def evaluate(root_node, color_set):
    c, e = self.evaluate_score_for_colors(
      root_node)
    return self.evaluate_cost(color_set, c, e)

  def evaluate_score_for_colors(root_node):
    inspected_edges, c, e = [], {}, {}

    for node in root_node.iterator():
      if node.color not in c:
        c[node.color] = 0

      c[node.color] += 1

      for child_node in node.edges:
        if {node, child_node} not in 
            inspected_edges and 
            color == child_node.color:
          if node.color not in e:
            e[node.color] = 0

          e[node.color] += 1
          
          inspected_edges.append(
            {node, child_node})

    return c, e
    
  def evaluate_cost(color_set, c, e):
    cost = 0

    for color in color_set:
      c_i, e_i = 0, 0

      if color in c:
        c_i = c[color]
      if color in e:
        e_i = e[color]

      cost += -1 * c_i ** 2 + 2 * c_i * e_i

    return cost
                    \end{verbatim}
                \end{minipage}
                
                &
        
                \begin{minipage}[t]{0.45\textwidth}
                    \noindent Metoda \verb+evaluate+ oblicza wartość funkcji kosztu dla danego grafu. Algorytm wykonywany jest w~dwóch krokach. \\ \\ \\
                    
                    \noindent W~pierwszym kroku obliczane są wartości $C_{i}$ oraz $E_{i}$ dla każdego koloru. Metoda \verb+evaluate_score_for_colors+ wykonuje niezbędne obliczenia. Istotne jest, że wszystkie wartości wyznaczane są w~czasie pojedynczego przejścia przez graf, dzięki czemu metoda jest wydajna. \\ \\ \\ \\ \\ \\ \\ \\ \\ \\ \\ \\ \\ \\ \\ \\
                    
                    \noindent Następnie zliczane są wyniki dla wszystkich kolorów znajdujących się w~zbiorze. Funkcja \verb+evaluate_cost+ oblicza wartość na podstawie wzoru $f(G) = -\sum_{i=1}^{k} C_i^2 + \sum_{i=1}^{k} 2 C_i E_i$, gdzie $C_{i}$ oznacza liczbę wierzchołków o~kolorze $i$, natomiast $E_{i}$ oznacza liczbę krawędzi, która łączy dwa wierzchołki o~kolorze $i$. \\
                \end{minipage}
            
                \\
            
            \end{tabular}
        
        \end{table}
        
Ponadto klasa \verb+CostEvaluator+ posiada metodę \verb+evaluate_score_for_permutation+. Pozwala ona na szybkie obliczanie funkcji celu dla permutacji pokolorowania grafu. Metoda przyjmuje parametry:

\begin{itemize}
    \item \verb+node+ wierzchołek, którego kolorowanie ulegnie zmianie w~trakcie permutacji.
    \item \verb+target_color+ docelowy kolor dla wierzchołka (po permutacji).
    \item \verb+base_c+ słownik wartości $C_{i}$ przed wykonaniem permutacji.
    \item \verb+base_e+ słownik wartości $E_{i}$ przed wykonaniem permutacji.
    \item \verb+color_set+ zbiór wszystkich kolorów.
\end{itemize}

\noindent Korzystając z~powyższych parametrów metoda wyznacza funkcję kosztu dokonując pojedynczego przejścia po wierzchołku oraz wszystkich jego sąsiadach, a~nie po całym grafie. Pozwala to znacząco zredukować czas szacowania funkcji kosztu dla permutacji.


\end{document}
