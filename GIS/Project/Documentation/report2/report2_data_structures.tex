\section{Struktury danych}
Podstawową strukturą danych użytą w programie jest graf, na który składa się zbiór wierchołków.
Pojedynczy wierzchołek zawiera:
\begin{itemize}
 \item identyfikator wierzchołka (ciąg znaków),
 \item identyfikator przypisanego koloru (liczba całkowita),
 \item zbiór wskazań na sąsiadujące wierzchołki (kolekcja wskazań).
\end{itemize}
