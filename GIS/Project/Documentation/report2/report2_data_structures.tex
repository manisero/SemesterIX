\section{Struktury danych}
\subsection{Graf}
Podstawową strukturą danych użytą w programie jest graf, na który składa się zbiór wierchołków.
Pojednynczy wierchołek zawiera:
\begin{itemize}
 \item identyfikator wierzchołka (ciąg znaków),
 \item identyfikator przypisanego koloru (liczba całkowita),
 \item zbiór wskazań na sąsiadujące wierzchołki (kolekcja wskazań).
\end{itemize}

\subsection{Pamięć}
Pamięć to listowa struktura danych, do której kolejne wpisy dodawane są na początku listy. W~pamięci przechowywane są pary (\emph{identyfikator wierzchołka}, \emph{kolor wierzchołka}).

\begin{itemize}
 \item Pamięć długoterminowa jest realizowana poprzez przechowywanie na liście wszystkich przejść od początku działania algorytmu.
 \item Pamięć krótkoterminowa jest realizowana poprzez użycie okna o~rozmiarze $m$, umieszczonego na początku listy.
\end{itemize}
