\section{Projekty testów}
Przetestowane zostaną następujące aspekty programu:

\subsection{Poprawność działania}
\label{sec:algorithm_correctness}
Poprawność działania programu zostanie zweryfikowana:

\begin{description}
 \item [Przy pomocy testów jednostkowych] Każda metoda wykorzystywana w~algorytmie zostanie przetestowana jednostkowo.
 \item [Przy pomocy zbiorów danych zaczerpniętych z~Internetu] Do testowania aplikacji zostaną wykorzystane znane optymalne pokolorowania grafu zaczerpnięte ze stron internetowych. Przykładowe dane znajdują się na serwerze \url{http://mat.gsia.cmu.edu/COLOR/instances.html}.
\end{description}

\subsection{Optymalny rozmiar pamięci tabu}
Optymalny rozmiar pamięci będziemy wyznaczać korzystając ze zbiorów danych testowych wymienionych w~sekcji \ref{sec:algorithm_correctness}. Będziemy porównywać czas wyznaczania oraz poprawność danego rozwiązania dla różnych rodzajów pamięci. 

\subsection{Wydajność}
\begin{itemize}
 \item Wydajność algorytmu zostanie zmierzona z użyciem zestawu grafów testowych różniących się liczbą wierzchołków i krawędzi.
 \item Dla celów testowych program zostanie zmodyfikowany tak, aby jako dane wejściowe przyjmował wstępnie pokolorowany graf.
  Pozwoli to:
  \begin{itemize}
   \item pominąć etap przygotowania grafu (patrz pkt. \ref{sec:exmpl_graph_prep}), a tym samym osiągnąć deterministyczne działanie programu,
   \item zbadać wpływ wstępnego pokolorowania na wydajność algorytmu.
  \end{itemize}
 \item Uzyskane wyniki zostaną ujęte w sprawozdaniu końcowym.
\end{itemize}
