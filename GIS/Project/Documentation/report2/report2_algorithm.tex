\section{Opis algorytmu}
Zadaniem programu jest pokolorowanie wierzchołków zadanego grafu z użyciem jak najmniejszej liczby kolorów.
Kolorowanie odbywa się z wykorzystaniem heurystycznego algorytmu przeszukiwania z tabu.
Węzłem przestrzeni przeszukiwań jest pokolorowany (legalnie bądź nie) graf.

\subsection{Sąsiedztwo}
Sąsiadami w przestrzeni przeszukiwań są takie dwa pokolorowane grafy $G$~i~$H$, że graf~$H$ można osiągnąć poprzez zmianę koloru jednego z wierzchołków grafu~$G$.

\subsection{Funkcja celu}
Algorytm dąży do minimalizacji funkcji celu\footnote{Definicja funkcji celu zaczerpnięta z: D. S. Johnson, C. R. Aragon, L. A. McGeoch, C. Schevon, Optimization by Simulated Annealing: An Experimental Evaluation; Part II, Graph Coloring and Number Partitioning, Operations Research, Vol. 39, No. 3, May-June 1991, pp. 378-406.}:

\begin{equation}
 f(G) = -\sum_{i=1}^{k} C_i^2 + \sum_{i=1}^{k} 2 C_i E_i
\end{equation}

gdzie:
\begin{itemize}
 \item $G$ - graf, dla którego liczona jest funkcja celu,
 \item $k$ - liczba kolorów użytych do pokolorowania grafu $G$,
 \item $C_i$ - liczba wierzchołków grafu $G$ pokolorowanych na $i$-ty kolor,
 \item $E_i$ - liczba krawędzi grafu $G$, których oba końce pokolorowane są na $i$-ty kolor.
\end{itemize}

Definicję funkcji należy rozumieć następująco:

\begin{enumerate}
 \item z jednej strony, faworyzowane są pokolorowania z użyciem jak najmniejszej liczby kolorów,
 \item z drugiej strony, dyskryminowane są pokolorowania nielegalne.
\end{enumerate}

\subsection{Lista tabu}
Lista tabu zawiera ograniczoną liczbę ostatnich akcji podjętych przez algorytm.
Pojedynczą akcją jest wybór wierzchołka, który zostanie pokolorowany na inny kolor.
Akcja na liście tabu jest reprezentowana przez parę \emph{identyfikator wierzchołka} oraz \emph{kolor wierzchołka przed podjęciem akcji}. 
Reprezentacja ta zapobiega badaniu jednakowych kombinacji w~kolejnych iteracjach algorytmu. Przykładowo, w~minimum lokalnym może zdarzyć się sytuacja, gdy najlepsze wartości funkcji celu będziemy osiągać poprzez cykliczną zmianę koloru tego samego wierzchołka w~kolejnych iteracjach, czego chcemy uniknąć.

\subsection{Pamięci}

W~algorytmie zostaną wykorzystane dwie pamięci:

\begin{description}
 \item [Krótkoterminowa] Realizuje listę tabu. Jej rozmiar jest parametrem algorytmu.
 \item [Długoterminowa] Przechowuje historię wszystkich akcji. Jest wykorzystywana podczas podejmowania decyzji o~wyborze najlepszej permutacji w~danej iteracji, w~momencie gdy istnieje wiele najlepszych permutacji o~jednakowej wartości funkcji celu.
\end{description}
