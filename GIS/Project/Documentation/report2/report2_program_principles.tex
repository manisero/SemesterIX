\section{Założenia programu}

\subsection{Złożoność obliczeniowa}

Złożoność obliczeniową można oszacować wyrażeniem:

\begin{equation}
	i * (nk * m + nk * n^2 + (nk * log(nk))
\end{equation}

\noindent gdzie:

\begin{itemize}
 \item $i$ - liczba iteracji,
 \item $n$ - liczba wierzchołków grafu,
 \item $m$ - rozmiar pamięci tabu.
\end{itemize}

\noindent Każda z~$i$ iteracji składa się z~następujących kroków:

\begin{enumerate}
 \item Wyznaczenie wszystkich możliwych sąsiadów aktualnego grafu, wraz ze sprawdzeniem dopuszczalności (lista tabu): $nk * m$.
 \item Obliczenie funkcji celu dla każdego dopuszczalnego sąsiada: $nk * n^2$. (Obliczenie funkcji celu wymaga odwiedzenia wszystkich wierzchołków oraz wszystkich wierzchołków z~nimi połączonych: $n^2$.)
 \item Sortowanie wyznaczonych sąsiadów ze względu na wartość funkcji celu: $nk * log(nk)$.
\end{enumerate}

\noindent Algorytm ma zatem złożoność wielomianową. 

\subsection{Dane wejściowe}

Na rysunku \ref{fig:input_data} przedstawiono przykładowe dane wejściowe. Dane wejściowe składają się z:

\begin{figure}[ht!]
	\begin{Verbatim}[frame=single]
3
A,B
C,D
A,D
C,E 
	\end{Verbatim}
	\caption{Przykładowe dane wejściowe.}
	\label{fig:input_data}
\end{figure}

\begin{description}
 \item [Liczebności klas kolorów] specyfikowanej jako pierwszy parametr. 
 \item [Połączeń pomiędzy wierzchołkami] specyfikowanych w kolejnych wierszach jako para identyfikatorów wierzchołków oddzielonych przecinkami.
\end{description}

\subsection{Dane wyjściowe}

Dane wyjściowe zależą od wybranego trybu programu (opisane w~sekcji \ref{sec:options}). 

\begin{description}
 \item [Tryb standardowy] Na wyjściu programu wypisywane są:
	\begin{itemize}
	 \item Nazwa pliku z~analizowanym grafem.
	 \item Wynikowe przyporządkowanie \emph{wierzchołek} $\leftrightarrow$ \emph{kolor}.
	 \item Sprawdzenie legalności wynikowego przyporządkowania (tak / nie).
	\end{itemize}
 \item [Tryb ,,rozmowny''] Na wyjściu programu, poza informacjami z~trybu standardowego, wypisywane są:
	\begin{itemize}
	 \item Dane początkowe - nazwy klas kolorów, wylosowane przyporządkowanie, wartość funkcji celu dla wylosowanego przyporządkowania.
	 \item Dane związane z~każdą iteracją - numer iteracji, liczba iteracji bez zmiany wyniku, wartość funkcji celu dla najlepszej permutacji w~iteracji, wartość funkcji celu dla najlepszego wyniku, zawartość pamięci krótkotrwałej dla danej iteracji, pokolorowanie (permutację) wybrane w~danej iteracji.
	\end{itemize}
\end{description}

\subsection{Parametry algorytmu}

\paragraph{Wielkość pamięci}

Rozmiar tablicy tabu jest wymaganym parametrem aplikacji. Rozmiar tablicy tabu jest specyfikowany opcją \verb+-m+. 

Przykład: \verb+<nazwa_programu> -m 5+ uruchamia algorytm z~pięcioelementową tablicą tabu.

\paragraph{Maksymalna liczba iteracji}

Maksymalna liczba iteracji określa liczbę przejść algorytmu, po której aplikacja zakończy działanie (opisane w~sekcji \ref{sec:stop_criteria}). Maksymalną liczbę iteracji specyfikuje się opcją \verb+-i+.

Przykład: \verb+<nazwa_programu> -i 500+ uruchamia algorytm dla maksymalnie 500 iteracji.

\paragraph{Maksymalna liczba iteracji bez zmiany rezultatu}

Maksymalna liczba iteracji bez zmiany rezultatu określa liczbę przejść algorytmu, po której aplikacja wyłączy się, jeżeli wartość funkcji celu dla najlepszego dotychczas znalezionego pokolorowania nie zmieni się (opisane w~sekcji \ref{sec:stop_criteria}). Maksymalną liczbę iteracji bez zmiany wyniku specyfikuje się opcją \verb+-s+.

Przykład: \verb+<nazwa_programu> -s 25+ uruchamia algorytm dla maksymalnie 25 iteracji bez zmiany wyniku.

\subsection{Opcje programu}
\label{sec:options}

Poza parametrami algorytmu, aplikacja udostępnia opisane poniżej opcje.

\paragraph{Plik(i) wejściowe}

W~opcjach programu można wyspecyfikować jeden lub więcej plików wejściowych. Nazwę pliku wejściowego specyfikuje się bez dodatkowych opcji, zaraz po nazwie programu.

Przykład: \verb+<nazwa_programu> graf1.txt graf2.txt+ wykona algorytm dla grafów opisanych w~plikach \emph{graf1.txt} oraz \emph{graf2.txt}.

\paragraph{Plik wyjściowy}

W~opcjach programu można wyspecyfikować nazwę pliku, do którego zostanie zapisane wyjście programu. Nazwę pliku wyjściowego specyfikuje się opcją \verb+-o+. Nazwa pliku wyjściowego jest parametrem opcjonalnym. Domyślnie wyjście przekierowywane jest na standardowy strumień (konsolę).

Przykład: \verb+<nazwa_programu> -o wyjscie1.txt+ zapisze wyjście algorytmu do pliku \emph{wyjscie1.txt}.

\paragraph{Tryb ,,rozmowny''}

W~opcjach programu można włączyć tryb ,,rozmowny'' (\emph{verbose}), który wyprowadza dodatkowe informacje diagnostyczne na wyjście w~trakcie działania algorytmu. W~trybie domyślnym na wyjście wyprowadzany jest tylko wynik działania algorytmu. Tryb ,,rozmowny'' specyfikuje się opcją \verb+-v+.

Przykład: \verb+<nazwa_programu> -v+ uruchamia aplikację w~trybie ,,rozmownym''.

\subsection{Kryteria stopu}
\label{sec:stop_criteria}

\paragraph{Maksymalna liczba iteracji}

Wykonywanie programu zakończy się, gdy algorytm przekroczy maksymalną liczbę iteracji. Maksymalna liczba iteracji jest podana jako parametr aplikacji.

\paragraph{Maksymalna liczba iteracji bez zmiany rezultatu}

Wykonywanie programu zakończy się, gdy algorytm przekroczy maksymalną liczbę iteracji, w~których nie zmieniła się wartość funkcji celu dla najlepszego pokolorowania. Maksymalna liczba iteracji bez zmiany wyniku jest podawana w parametrach aplikacji.

\paragraph{}
W obu przypadkach jako wynik działania programu zostanie podane najlepsze dotychczas znalezione pokolorowanie badanego grafu.


\subsection{Sytuacje wyjątkowe}

\paragraph{Brak któregokolwiek z wymaganych parametrów programu}
W tym przypadku działanie programu kończy się niepowodzeniem, a na wyjście podawana jest informacja o poprawnym sposobie wywołania.

\paragraph{Niepoprawny format pliku wejściowego}
W tym przypadku działanie programu kończy się niepowodzeniem, a na wyjście podawana jest informacja o błędnym formacie pliku.

\paragraph{Niespójny graf wejściowy}
W tym przypadku działanie programu kończy się niepowodzeniem, a na wyjście podawana jest informacja o błędnej konstrukcji grafu.

\paragraph{Brak możliwości wykonania jakiejkolwiek akcji}
Ta sytuacja może zdarzyć się w przypadku, gdy lista tabu ma zbyt duży rozmiar i zawiera wszystkie możliwe w danym kroku akcje (tzn. każda możliwa permutacja aktualnego grafu jest zabroniona).
Aby umożliwić dalsze działanie, program usuwa najstarsze elementy listy, dopóki nie osiągnie stanu, w którym dozwolona będzie przynajmniej jedna z możliwych akcji.