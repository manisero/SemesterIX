\documentclass[a4paper,10pt]{article}

\usepackage{polski}
\usepackage[utf8]{inputenc}

\title{Grafy i Sieci. Sprawozdanie 1.}
\author{Aniserowicz Michał, Turek Jakub}

\begin{document}

\maketitle

\section{Temat projektu}

SK11 Kolorowanie grafu za pomocą przeszukiwania z tabu.



\section{Interpretacja tematu}
Zadaniem programu powstałego w wyniku realizacji projektu będzie pokolorowanie wierzchołków zadanego grafu z użyciem jak najmniejszej liczby kolorów.
Kolorowanie odbywać się będzie z wykorzystaniem heurystycznego algorytmu przeszukiwania z tabu.
Węzłem przestrzeni przeszukiwań będzie pokolorowany (legalnie bądź nie) graf.

\subsection{Funckja celu}
Algorytm będzie dążył do minimalizacji funkcji celu\footnote{Definicja funkcji celu zaczerpnięta z: D. S. Johnson, C. R. Aragon, L. A. McGeoch, C. Schevon, Optimization by Simulated Annealing: An Experimental Evaluation; Part II, Graph Coloring and Number Partitioning, Operations Research, Vol. 39, No. 3, May-June 1991, pp. 378-406.}:

\begin{equation}
 f(G) = -\sum_{i=1}^{k} C_i^2 + \sum_{i=1}^{k} 2 C_i E_i
\end{equation}

gdzie:
\begin{itemize}
 \item $G$ - graf, dla którego liczona jest funkcja celu,
 \item $k$ - liczba kolorów użytych do pokolorowania grafu $G$,
 \item $C_i$ - liczba wierzchołków grafu $G$ pokolorowanych na $i$-ty kolor,
 \item $E_i$ - liczba krawędzi grafu $G$, których oba końce pokolorowane są na $i$-ty kolor.
\end{itemize}

Definicję funkcji należy rozumieć następująco:

\begin{enumerate}
 \item z jednej strony, faworyzowane są pokolorowania z użyciem jak najmniejszej liczby kolorów,
 \item z drugiej strony, dyskryminowane są pokolorowania nielegalne.
\end{enumerate}

\subsection{Lista tabu}
Lista tabu zawierać będzie ograniczoną liczbę ostatnich akcji podjętych przez algorytm.
Pojedynczą akcją będzie pokolorowanie pojedycznego wierzchołka na określony kolor.


\section{Dane wejściowe i wyjściowe}
Program przyjmował będzie ścieżkę do pliku zawierającego definicję grafu, np.:

\begin{verbatim}
A
B
C

A,B
B,C
\end{verbatim}

Plik wynikowy zawierał będzie listę par wierzchołek-kolor, np.:

\begin{verbatim}
A,1
B,2
C,1
\end{verbatim}



\section{Sposób testowania programu}

\begin{itemize}
 \item Poprawność działania programu zostanie zweryfikowana przy pomocy testów jednostkowych.
 \item Wydajność algorytmu zostanie zmierzona z użyciem zestawu grafów testowych różniących się liczbą wierzchołków i krawędzi.
\end{itemize}







\section{Założenia niefunkcjonalne}

\begin{itemize}
 \item Język programowania: Python.
 \item Postać wynikowego programu: aplikacja konsolowa.
\end{itemize}


\end{document}
