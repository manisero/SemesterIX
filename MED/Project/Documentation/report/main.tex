\documentclass[a4paper,10pt]{article}
\usepackage[utf8]{inputenc}
\usepackage[MeX]{polski}
\usepackage{tikz}
\usepackage{url}
\usepackage{fullpage}

\title{Projekt MED-P3, algorytm GRM. Raport. \\ \small{Przedmiot: Metody eksploracji danych w odkrywaniu wiedzy.}}
\author{Michał Aniserowicz, Jakub Turek}
\date{}

\begin{document}

\maketitle

\section{Opis zadania}
Celem projektu jest zaimplementowanie algorytmu wyznaczania reguł decyzyjnych o minimalnych poprzednikach, które są częstymi generatorami.
Algorytm ten jest modyfikacją algorytmu odkrywania częstych generatorów (GRM), opisanego w~\cite{grm}.



\section{Założenia}
poczynione zalozenia
- kazda transakcja bedzie miala decyzje
- aplikacja konsolowa pobieracjaca dane z pliku i zwracajaca wynik w dwoch formatach
- aplikacja mierzy czas poszczegolnych krokow
- C\#, .NET 3,5



\section{Dane wejściowe i wyjściowe}
opis danych wejsciowych i wyjsciowych
- opcje (zostana opisane pozniej), minsup bezwzgledne!
- dane oddzielone przecinkami (decyzja razem z atrybutami, na dowolnym miejscu)
- naglowki w pierwszym wierszu
- brak danych - spacja (biale znaki)
- dwa formaty wynikow
- oprocz tego wynik na konsoli



\section{Implementacja}
wszystkie istotne kwestie zwiazne z projektowaniem (np. diagramy klas) i implementacja
projektowanie:
- podzial na moduly (console, dataset processing, GRM)
- testy
- diagram klas Logic
implementacja:
- jakis algorytm, moze z diffsetami
- rozne sortowania
- tidset/diffset
- bruteforce/inv list
- tracking (poziomy)

\subsection{Opymalizacje}
- wszystkie wartosci otrzymuja identyfikatory liczbowe
- skonfliktowane generatory
- transaction ids - posortowane (szybkie intersect, except)

roznice z GRM:
- dany node jest decyzyjny - nie rozwijamy go (bo generatory dzieci nie beda minimalne)
- generatory decyzji trzymane w slowniku (klucz - decyzja), posortowane wg hasha
- w ogole nie ma granicy
- dla diffsetow transaction ids trzymane w slowniku (klucz - decyzja)



\section{Podręcznik użytkownika}
podrecznik potencjalnego uzytkownika wytworzonego oprogramowania
(zamierzam korzystać z niego podczas sprawdzania Panstwa rozwiazan)
- wszystkie opcje programu
- przykladowa komenda i wynik na konsoli



\section{Analiza poprawności}
wszystkie wyniki wytwarzane przez program otrzymane dla malego,
przykladowego zbioru danych (w celu weryfikacji poprawnosci działania
programu)
- przyklad z konsultacji



\section{Analiza wydajności}
wyniki jakosciowe i ilosciowe na (np. czas dzialania; liczba wzorcow)
uzyskane dla wiekszych (wielkich) zbiorow danych(np. z
http://archive.ics.uci.edu/ml/ or http://fimi.cs.helsinki.fi/data/ lub
uzgodnionych już wcześniej ze mna podczas konsultacji projektowych)
- wykresy, wykresy
- ze dla duzej liczby atrybutow malo wydajny



\section{Wnioski}
wnioski z realizacji projektu
- ze trzeba by poprawic wykrywanie supergeneratorow
- ze ogolnie dziala spoczko (nursey)



\begin{thebibliography}{*}

 \bibitem{grm}
  \emph{Odkrywanie reprezentacji generatorowej wzorców częstych z wykorzystaniem struktur listowych},
  Kryszkiewicz M., Pielasa P.,
  Instytut Informatyki,
  Politechnika Warszawska.


\end{thebibliography}

\end{document}
