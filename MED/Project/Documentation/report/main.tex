\documentclass[a4paper,10pt]{article}
\usepackage[utf8]{inputenc}
\usepackage[MeX]{polski}
\usepackage{tikz}
\usepackage{url}
\usepackage{fullpage}

\title{Projekt MED-P3, algorytm GRM. Raport. \\ \small{Przedmiot: Metody eksploracji danych w odkrywaniu wiedzy.}}
\author{Michał Aniserowicz, Jakub Turek}
\date{}

\begin{document}

\maketitle

\section{Opis zadania} \label{sec:task_desc}
Celem projektu jest zaimplementowanie algorytmu wyznaczania reguł decyzyjnych o minimalnych poprzednikach, które są częstymi generatorami.
Algorytm ten jest modyfikacją algorytmu odkrywania częstych generatorów (GRM), opisanego w~\cite{grm}.



\section{Założenia} \label{sec:assumptions}
Projekt zrealizowano w oparciu o następujące założenia:

 \subsection{Niefunkcjonalne:}
  \begin{enumerate}
   \item Użyty język programowania; platforma: C\#; .NET Framework 3.5.
   \item Obsługiwane systemy operacyjne: kompatybilne z .NET Framework 3.5\footnote{Lista systemów kompatybilnych z .NET Framework 3.5 dostępna jest pod adresem: \emph{http://msdn.microsoft.com/en-us/library/vstudio/bb882520\%28v=vs.90\%29.aspx}, sekcja ``Supported Operating Systems''.} (aplikację testowano na systemie Microsoft Windows 7 Ultimate).
   \item Rodzaj aplikacji: aplikacja konsolowa uruchamiana z wiersza poleceń.
 \end{enumerate}


 \subsection{Funkcjonalne:}
  \begin{enumerate}
   \item Aplikacja pobiera dane z pliku (patrz sekcja~\ref{sec:input:file}).
   \item Aplikacja zwraca wynik działania w dwóch formatach: ``przyjaznym dla człowieka'' i ``excelowym'' (patrz sekcja~\ref{sec:output}).
   \item Aplikacja pozwala mierzyć czas wykonania poszczególnych kroków algorytmu.
  \end{enumerate}


 \subsection{Dotyczące danych wejściowych:} \label{sec:assumptions:input}
  \begin{enumerate}
   \item Dane wejściowe zawierają jedynie wartości atrybutów transakcji i ewentualnie nazwy atrybutów transakcji.
   \item Wartości atrybutów w pliku wejściowym są oddzielone przecinkami.
   \item Każda transakcja ma przypisaną decyzję.
   \item Brakujące wartości atrybutów (tzn. wartości nieznane bądź nieustalone) są oznaczone jako wartości puste lub złożone z białych znaków.
  \end{enumerate}



\section{Dane wejściowe} \label{sec:input}
Aplikacja będąca wynikiem projektu przyjmuje jednocześnie dwa rodzaje danych wejściowych:

\begin{itemize}
 \item plik zawierający dane transakcji,
 \item parametry podane przez użytkownika w wierszu poleceń.
\end{itemize}

 \subsection{Plik wejściowy} \label{sec:input:file}
 Plik wejściowy powinien zawierać kolejne wartości atrybutów transakcji według reguł przedstawionych w sekcji~\ref{sec:assumptions:input}.
 Przykładowy format pliku wejściowego:
 
\begin{verbatim}
a,b,c,d,e, ,g, ,+
a,b,c,d,e,f, , ,+
a,b,c,d,e, , ,h,+
a,b, ,d,e, , , ,+
a, ,c,d,e, , ,h,-
 ,b,c, ,e, , , ,-
\end{verbatim}

 Opcjonalnie, pierwszy wiersz pliku wejściowego może zawierać nazwy atrybutów transakcji (nagłówki), na przykład:

\begin{verbatim}
Attr A,Attr B,Attr C,Attr D,Attr E,Attr F,Attr G,Attr H,Decision
a,b,c,d,e, ,g, ,+
 ,b,c, ,e, , , ,-
\end{verbatim}

 Jeden z atrybutów transakcji musi reprezentować przypisaną jej decyzję.
 Domyślnie, aplikacja uznaje ostatni atrybut transakcji za ``decyzyjny'' - użytkownik może jednak samodzielnie wskazać odpowiedni atrybut (patrz sekcja~\ref{sec:input:cmd}).
 
 
 \subsection{Parametry wiersza poleceń} \label{sec:input:cmd}
 Aplikacja przyjmuje następujące parametry:
 
 \begin{itemize}
  \item \verb+--help+ - Powoduje wyświetlenie informacji o dostępnych parametrach i wyjście z programu.
  \item \verb+-f, --file=VALUE+ - Ścieżka do pliku wejściowego. Parametr wymagany.
  \item \verb+--sup, --minSup=VALUE+ - Próg (bezwzględny) wsparcia - wykryte zostaną reguły decyzyjne o poprzednikach cechujących się wsparciem większym \textbf{lub równym} progowi wsparcia. Parametr wymagany.
  \item \verb+-h, --headers+ - Flaga oznaczająca, że plik wejściowy zawiera nagłówki atrybutów. Parametr opcjonalny.
  \item \verb+--dec, --decAttr=VALUE+ - Pozycja atrybutu zawierającego wartości decyzji (\verb+1+ - pierwszy atrybut, \verb+2+ - drugi atrybut itd.). Parametr opcjonalny (jeśli nie zostanie podany, ostatni atrybut zostanie uznany za decyzyjny).
  \item \verb+--sort=VALUE+ - strategia sortowania elementów (patrz sekcja~\ref{sec:impl:sort}). Parametr opcjonalny.
  Dopuszczalne wartości:
  \begin{itemize}
   \item \verb+AscendingSupport+ (lub \verb+0+; wartość domyślna),
   \item \verb+DescendingSupport+ (lub \verb+1+),
   \item \verb+Lexicographical+ (lub \verb+2+).
  \end{itemize}
 
  \item \verb+--store=VALUE+ - strategia przechowywania identyfikatorów transakcji (patrz sekcja~\ref{sec:impl:store}). Parametr opcjonalny.
  Dopuszczalne wartości:
  \begin{itemize}
   \item \verb+TIDSets+ (lub \verb+0+; wartość domyślna),
   \item \verb+DiffSets+ (lub \verb+1+).
  \end{itemize}
  
  \item \verb+--supgen=VALUE+ - Strategia przechowywania generatorów decyzji, a także wykrywania i usuwania ich nadgeneratorów (patrz sekcja~\ref{sec:impl:supgen}). Parametr opcjonalny.
  Dopuszczalne wartości:
  \begin{itemize}
   \item \verb+InvertedLists+ (lub \verb+0+; wartość domyślna),
   \item \verb+BruteForce+ (lub \verb+1+).
  \end{itemize}
  
  \item \verb+--track=VALUE+ - Poziom monitorowania wydajności programu (patrz sekcja~\ref{sec:impl:track}). Parametr opcjonalny.
  \begin{itemize}
   \item \verb+NoTracking+ (lub \verb+0+),
   \item \verb+Task+ (lub \verb+1+; wartość domyślna).
   \item \verb+Steps+ (lub \verb+2+).
   \item \verb+Substeps+ (lub \verb+3+; Uwaga: może powodować znaczący spadek ogólnej wydajności programu).
  \end{itemize}
  
  \item \verb+-o, --output=VALUE+ - Ścieżka plików wyjściowych. Parametr opcjonalny. Poprawna wartość jest ścieżką pliku z pominięciem jego rozszerzenia (np. \emph{wyniki/wynik}. Wartość domyślna: \emph{[ścieżka pliku wejściowego]\_rules}.
 \end{itemize}



\section{Dane wyjściowe} \label{sec:output}
opis danych wyjsciowych
- dwa formaty wynikow
- oprocz tego wynik na konsoli



\section{Implementacja} \label{sec:impl}
wszystkie istotne kwestie zwiazne z projektowaniem (np. diagramy klas) i implementacja
projektowanie:
- podzial na moduly (console, dataset processing, GRM)
- testy
- diagram klas Logic
implementacja:
- jakis algorytm, moze z diffsetami
- rozne sortowania
- tidset/diffset
- bruteforce/inv list
- tracking (poziomy)
biblioteki zewnętrzne:
- NDeskOptions
- xunit
- moq

 \subsection{Strategie sortowania elementów} \label{sec:impl:sort}
 
 \subsection{Strategie przechowywania identyfikatorów transakcji} \label{sec:impl:store}
 
 \subsection{Strategie przechowywania generatorów decyzji} \label{sec:impl:supgen}
 
 \subsection{Monitorowanie wydajności programu} \label{sec:impl:track}

 \subsection{Opymalizacje}
 - wszystkie wartosci otrzymuja identyfikatory liczbowe
 - skonfliktowane generatory
 - transaction ids - posortowane (szybkie intersect, except)

 roznice z GRM:
 - dany node jest decyzyjny - nie rozwijamy go (bo generatory dzieci nie beda minimalne)
 - generatory decyzji trzymane w slowniku (klucz - decyzja), posortowane wg hasha
 - w ogole nie ma granicy
 - dla diffsetow transaction ids trzymane w slowniku (klucz - decyzja)



\section{Podręcznik użytkownika}
podrecznik potencjalnego uzytkownika wytworzonego oprogramowania
(zamierzam korzystać z niego podczas sprawdzania Panstwa rozwiazan)
- wszystkie opcje programu
- przykladowa komenda i wynik na konsoli



\section{Analiza poprawności}
wszystkie wyniki wytwarzane przez program otrzymane dla malego,
przykladowego zbioru danych (w celu weryfikacji poprawnosci działania
programu)
- przyklad z konsultacji



\section{Analiza wydajności}
wyniki jakosciowe i ilosciowe na (np. czas dzialania; liczba wzorcow)
uzyskane dla wiekszych (wielkich) zbiorow danych(np. z
http://archive.ics.uci.edu/ml/ or http://fimi.cs.helsinki.fi/data/ lub
uzgodnionych już wcześniej ze mna podczas konsultacji projektowych)
- uzasadnic supergenerators Inverted Lists
- ze sortowanie AscendingSupport
- ze storage TIDSets
- wykresy, wykresy
- ze dla duzej liczby atrybutow malo wydajny



\section{Wnioski}
wnioski z realizacji projektu
- ze trzeba by poprawic wykrywanie supergeneratorow
- ze sortowanie ma duzy wplyw
- ze ogolnie dziala spoczko (nursey)



\begin{thebibliography}{*}

 \bibitem{grm}
  \emph{Odkrywanie reprezentacji generatorowej wzorców częstych z wykorzystaniem struktur listowych},
  Kryszkiewicz M., Pielasa P.,
  Instytut Informatyki,
  Politechnika Warszawska.


\end{thebibliography}

\end{document}
