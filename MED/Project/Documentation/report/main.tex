\documentclass[a4paper,10pt]{article}
\usepackage[utf8]{inputenc}
\usepackage[MeX]{polski}
\usepackage{tikz}
\usepackage{url}
\usepackage{fullpage}

\title{Projekt MED-P3, algorytm GRM. Raport. \\ \small{Przedmiot: Metody eksploracji danych w odkrywaniu wiedzy.}}
\author{Michał Aniserowicz, Jakub Turek}
\date{}

\begin{document}

\maketitle

\section*{Opis zadania}
Celem projektu jest zaimplementowanie algorytmu wyznaczania reguł decyzyjnych o minimalnych poprzednikach, które są częstymi generatorami.
Algorytm ten jest modyfikacją algorytmu odkrywania częstych generatorów (GRM), opisanego w~\cite{grm}.



\section*{Założenia}
poczynione zalozenia



\section*{Dane wejściowe i wyjściowe}
opis danych wejsciowych i wyjsciowych



\section*{Implementacja}
wszystkie istotne kwestie zwiazne z projektowaniem (np. diagramy klas) i implementacja



\section*{Podręcznik użytkownika}
podrecznik potencjalnego uzytkownika wytworzonego oprogramowania
(zamierzam korzystać z niego podczas sprawdzania Panstwa rozwiazan)



\section*{Analiza poprawności}
wszystkie wyniki wytwarzane przez program otrzymane dla malego,
przykladowego zbioru danych (w celu weryfikacji poprawnosci działania
programu)



\section*{Analiza wydajności}
wyniki jakosciowe i ilosciowe na (np. czas dzialania; liczba wzorcow)
uzyskane dla wiekszych (wielkich) zbiorow danych(np. z
http://archive.ics.uci.edu/ml/ or http://fimi.cs.helsinki.fi/data/ lub
uzgodnionych już wcześniej ze mna podczas konsultacji projektowych)



\section*{Wnioski}
wnioski z realizacji projektu



\begin{thebibliography}{*}

 \bibitem{grm}
  \emph{Odkrywanie reprezentacji generatorowej wzorców częstych z wykorzystaniem struktur listowych},
  Kryszkiewicz M., Pielasa P.,
  Instytut Informatyki,
  Politechnika Warszawska.


\end{thebibliography}

\end{document}
