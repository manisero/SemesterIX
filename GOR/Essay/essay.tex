\documentclass[12pt]{article}
\usepackage[utf8]{inputenc}
\usepackage{fullpage}
\usepackage{polski}
\usepackage{hyperref}

\title{[GOR] Esej z~makroekonomii}
\author{Michał Aniserowicz, Marek Lewandowski, Jakub Turek}
\date{}

\begin{document}
    \maketitle
    
    Praca jest to świadoma, celowa działalność człowieka służąca zaspokajaniu potrzeb lub polegająca na wytwarzaniu dóbr\footnote{Wielki słownik ortograficzny - PWN 2003, 2006, 2008 - E. Polański}. W~języku polskim słowo praca etymologicznie jest związane z~trudem, wysiłkiem, mozołem, utrapieniem\footnote{Bańkowski A.: Etymologiczny słownik języka polskiego, t.2, Wyd. Naukowe PWN, Warszawa 2000}. Praca jest przede wszystkim źródłem dochodu, umożliwia zaspokajanie potrzeb człowieka. Ponadto praca pozwala budować więzi społeczne, daje poczucie życiowego celu i~pozwala zaspokajać ludzkie ambicje. 
    
    Obecnie praca (pod pojęciem zatrudnienia) jest regulowana prawnie i~rozumiana jako umowa pomiędzy dwiema stronami - pracownikiem oraz pracodawcą. Pracodawca jest odpowiedzialny za realizację przedsięwzięcia, do którego pracownik wnosi wykonywaną pracę otrzymując w~zamian (przeważnie) wynagrodzenie pieniężne. Pierwsze ustawy prawne regulujące zatrudnienie powstały w~XIX wieku na fali rewolucji przemysłowej. Ruch ten był niezbędny ze względu na fatalne warunki, w~jakich wykonywana była praca. Początkowo ustawy zabezpieczały wyłącznie podstawowe prawa osób zatrudnionych, takie jak ograniczona liczba godzin pracy czy odpowiednie warunki pracy. Później uregulowano również prawa do udziału pracowników w~zarządzaniu przedsiębiorstwami, sprawy równouprawnienia grup pracowników, czy też zabezpieczenia na wypadek utraty pracy.
    
    Z~pojęciem zatrudnienia nieodłącznie związane jest bezrobocie. Według internetowego słownika języka polskiego PWN\footnote{\url{http://sjp.pwn.pl/slownik/2444051/bezrobocie}} bezrobocie to zjawisko braku pracy zawodowej. Według polskich przepisów osoba bezrobotna jest to osoba, która nie jest zatrudniona, nie prowadzi działalności gospodarczej i~nie wykonuje innej pracy zawodowej, jest zdolna i~gotowa podjąć zatrudnienie, a~w~szczególności\footnote{Ustawa o promocji zatrudnienia i instytucjach rynku pracy. Dz. U. z 2004 r. Nr 99, poz. 1001, Art. 2, punkt 2.}:
    
    \begin{itemize}
        \item Ukończyła 18 rok życia.
        \item Nie osiągnęła wieku emerytalnego.
        \item Aktualnie nie uczy się na żadnym szczeblu kształcenia lub nie jest skierowana na szkolenie przez PUP.
        \item Jest zameldowana lub pozostaje w Polsce legalnie lub jej pobyt może zostać zalegalizowany (azyl polityczny, karta stałego lub czasowego pobytu, obywatele UE).
    \end{itemize}
    
    Bezrobocie niesie ze sobą zarówno negatywne, jak również pozytywne skutki dla gospodarki. Negatywne skutki są oczywiste. Brak środków na zaspokajanie własnych potrzeb może prowadzić do problemów rodzinnych, zdrowotnych, obniżenia samooceny jednostki. Skutki pozytywne obejmują między innymi wzrost średniego wykształcenia społeczeństwa (konkurencja na rynku pracy prowadzi do konieczności stałego podnoszenia własnych kwalifikacji).
    
    Według danych Eurostatu\footnote{\url{http://epp.eurostat.ec.europa.eu/portal/page/portal/employment_unemployment_lfs/introduction}} odsetek bezrobocia w~Polsce wynosi obecnie 10.2\% (stan aktualny dla listopad 2013). Jest to odsetek porównywalny ze wskaźnikiem średniego bezrobocia w~Unii Europejskiej (10.8\% w~listopadzie 2013). W~przeciągu ostatnich 16 lat odsetek bezrobocia w~Polsce wahał się w~granicach od 6.8\% (wrzesień 2008) do 20.3\%. 
\end{document}
