\documentclass[12pt]{article}
\usepackage[utf8]{inputenc}
\usepackage{fullpage}
\usepackage{polski}
\usepackage{hyperref}

\linespread{1.5}

\title{Bezrobocie \\ \Large{Przyczyny, przeciwdziałanie, przewidywany wpływ nowelizacji ustaw na odsetek bezrobocia}}
\author{Michał Aniserowicz, Marcin Cieślikowski, Marek Lewandowski, Jakub Turek}
\date{}

\begin{document}
    \maketitle
    
    \section*{Wstęp}
    
    W~pracy został poruszony problem bezrobocia, ze szczególnym naciskiem na występowanie tego zjawiska w~Polsce. Autorzy przybliżają genezę bezrobocia, wychodząc od korzeni pojęcia pracy i~ustawodawstwa o zatrudnienu. Praca opisuje przyczyny powstawania bezrobocia, sposoby walki z~nim i~plusy oraz minusy każdego z~przedstawionych podejść. Autorzy przedstawiają też własne rozważania na temat wpływu największych poprawek wprowadzonych do prawa pracy w~ostatnich latach na odsetek bezrobocia w~Polsce.
    
    \section*{Bezrobocie - informacje ogólne}
    
    Praca jest to świadoma, celowa działalność człowieka służąca zaspokajaniu potrzeb lub polegająca na wytwarzaniu dóbr~\cite{ort}. W~języku polskim słowo praca etymologicznie jest związane z~trudem, wysiłkiem, mozołem, utrapieniem~\cite{etym}. Praca jest przede wszystkim źródłem dochodu, umożliwia zaspokajanie potrzeb człowieka. Ponadto praca pozwala budować więzi społeczne, daje poczucie życiowego celu i~pozwala zaspokajać ludzkie ambicje. 
    
    Obecnie praca (pod pojęciem zatrudnienia) jest regulowana prawnie i~rozumiana jako umowa pomiędzy dwiema stronami - pracownikiem oraz pracodawcą. Pracodawca jest odpowiedzialny za realizację przedsięwzięcia, do którego pracownik wnosi wykonywaną pracę otrzymując w~zamian (przeważnie) wynagrodzenie pieniężne. Pierwsze ustawy prawne regulujące zatrudnienie powstały w~XIX wieku na fali rewolucji przemysłowej. Ruch ten był niezbędny ze względu na fatalne warunki, w~jakich wykonywana była praca. Początkowo ustawy zabezpieczały wyłącznie podstawowe prawa osób zatrudnionych, takie jak ograniczona liczba godzin pracy czy odpowiednie warunki pracy. Później uregulowano również prawa do udziału pracowników w~zarządzaniu przedsiębiorstwami, sprawy równouprawnienia grup pracowników, czy też zabezpieczenia na wypadek utraty pracy.
    
    Z~pojęciem zatrudnienia nieodłącznie związane jest bezrobocie. Według internetowego słownika języka polskiego PWN~\cite{sjp} bezrobocie to zjawisko braku pracy zawodowej. Według polskich przepisów osoba bezrobotna jest to osoba, która nie jest zatrudniona, nie prowadzi działalności gospodarczej i~nie wykonuje innej pracy zawodowej, jest zdolna i~gotowa podjąć zatrudnienie, a~w~szczególności~\cite{ust:pro:zat}:
    
    \begin{itemize}
        \item Ukończyła 18 rok życia.
        \item Nie osiągnęła wieku emerytalnego.
        \item Aktualnie nie uczy się na żadnym szczeblu kształcenia lub nie jest skierowana na szkolenie przez PUP.
        \item Jest zameldowana lub pozostaje w Polsce legalnie lub jej pobyt może zostać zalegalizowany (azyl polityczny, karta stałego lub czasowego pobytu, obywatele UE).
    \end{itemize}
    
    Bezrobocie niesie ze sobą zarówno negatywne, jak również pozytywne skutki dla gospodarki. Negatywne skutki są oczywiste. Brak środków na zaspokajanie własnych potrzeb może prowadzić do problemów rodzinnych, zdrowotnych, obniżenia samooceny jednostki. Skutki pozytywne obejmują między innymi wzrost średniego wykształcenia społeczeństwa (konkurencja na rynku pracy prowadzi do konieczności stałego podnoszenia własnych kwalifikacji).
    
    Według danych Eurostatu~\cite{eurostat} odsetek bezrobocia w~Polsce wynosi obecnie 10.2\% (stan aktualny dla listopad 2013). Jest to odsetek porównywalny ze wskaźnikiem średniego bezrobocia w~Unii Europejskiej (10.8\% w~listopadzie 2013). W~przeciągu ostatnich 16 lat odsetek bezrobocia w~Polsce wahał się w~granicach od 6.8\% (wrzesień 2008) do 20.3\%. Powyższe dane uwzględniają również \emph{fikcyjnych bezrobotnych} czyli osoby, które pobierają świadczenia z~tytułu braku zatrudnienia, jednak nie są faktycznie osobami bezrobotnymi.
    
    \section*{Rodzaje bezrobocia}
    Bezrobocie jest wywołanie przez różne przyczyny. Bezrobocie wywyołane przez każdą z~możliwych przyczyn wymaga podjęcia różnych działań zaradczych. Z~tego powodu ekspercji dzielą bezrobocie na następujące rodzaje~\cite{eko}:
    \begin{description}
        \item[Bezrobocie przejściowe] Przyczyną tego bezrobocia jest przejście ludzi z~jednego miejsca pracy do drugiego. Oba miejsca pracy wymagają od pracowników podobnych kwalifikacji. Zwykle trwa od kilku dni do kliku tygodni.
        \item[Bezrobocie cykliczne] Bezrobocie to jest powiązane z~cyklicznymi spadkami koniunktury gospodarczej. Zwykle trwa przez okres kilku tygodni lub miesięcy, aż do wzrosty produkcji i popytu na siłę roboczą.
        \item[Bezrobocie strukturalne] Wywołane jest potrzebą zmiany kwalifikacji pracowników. Może być wywyołane rozwojem technologii, czynnikami geograficznymi lub demograficznymi, zmianami struktury popytu konsumpcyjnego lub kunkurencją zagraniczną. Może utrzymywać się latami, aż pracownicy uzyskają odpowiednie kwalifikacji do nowych miejsc pracy.
        \item[Bezrobocie ukryte] Obejmuję ludzi, którzy nie są ujęci w~statystykach. Spowodowane jest to tym, że pracują w niepełnych wymiarze pomimo wyrażenia chęci do pracy w~pełnym wymiarze. Dotyczy to również osób, którzy pracują na stanowiskach nie wymagających od nich posiadanych przez nich kwalifikacji. Bezrobotni mogą też wyrejestrowywać się z~urzędów pracy po długim czasie bezskutecznych prób szukania pracy.
    \end{description}

    \section*{Sposoby walki z~bezrobociem}
    
    Wyszczególniane są dwa rodzaje walki z~bezrobociem:
    
    \begin{description}
        \item[Aktywny] obejmuje tworzenie specjalnych programów, które umożliwiają zatrudnianie osób długotrwale bezrobotnych, finansowanie staży dla absolwentów uczelni wyższych, organizację szkoleń, które umożliwiają zdobycie dodatkowych kwalifikacji oraz inne działania, które bezpośrednio oddziałują na szansę zdobycia pracy przez osoby bezrobotne.
        \item[Pasywny] obejmuje formy walki ze skutkami bezrobocia, które nie przyczyniają się bezpośrednio do powstawania nowych miejsc pracy. Przykładem może być podwyższanie kwot zapomóg dla bezrobotnych.
    \end{description}
   
   Żadne z~przytoczonych sposobów nie gwarantują skutecznej redukcji bezrobocia. W~przypadku sposobów aktywnych problemem jest dotarcie do osób, które są rzeczywiście zainteresowane skorzystaniem z~tego typu pomocy. Szacuje się, że około 40\% osób zarejestrowanych jako bezrobotne w~rzeczywistości są zatrudnione w~,,szarej strefie''. Dzięki prawnemu statusowi bezrobotnemu nie tracą oni przywileju do korzystania z~bezpłatnej służby zdrowia. Z~tego powodu plany aktywizacji zawodowej długotrwałych bezrobotnych nie odnoszą zamierzonego efektu. Większość osób, które przez długi okres przebywają na bezrobociu to pracownicy ,,szarej strefy''.
   
   Problemem pasywnej walki z~bezrobociem jest nadmierne przyzwyczajenie obywateli do pomocy ze strony państwa. Wykorzystują to głównie osoby, którym bardziej opłaca się przebywać na zasiłku dla bezrobotnych niż podjąć pracę za najniższą krajową pensję. Sytuacja taka jest również kłopotliwa w~przypadku napływu imigrantów z~krajów, które są znacznie mniej rozwinięte gospodarczo, a~co za tym idzie siła nabywcza pieniądza jest w~nich znacznie większa. Osoby takie często nadmiernie eksploatują pomoc socjalną kraju, do którego emigrowały. Sytuacja taka została ostatnio nagłośniona w~związku z~wypowiedzią premiera Wielkiej Brytanii - Davida Camerona - o~imigrantach.
    
    \section*{Wpływ reform na wskaźnik zatrudnienia}
    
    W~tym rozdziale analizować będziemy możliwy wpływ ostatnich zmian w~ustawach uchwalonych przez Sejm Rzeczypospolitej Polskiej, które mogą mieć wpływ na sytuację zatrudnienia w Polsce.
        
    \paragraph{Zwiększenie płacy minimalnej} 
    
    Od pierwszego stycznia 2014 roku uległa zwiększeniu płaca minimalna w~Polsce. Została ona podwyższona o 80 zł do pułapu 1680 zł brutto wynagrodzenia za pracę w~pełnym wymiarze czasu. Według analiz firmy Sedlak \& Sedlak\footnote{\url{http://tiny.pl/q5nhj}} Polska zajmuje obecnie 12 miejsce pod względem wysokości płacy minimalnej wśród 21 krajów członkowskich Unii Europejskiej. Podwyższenie płacy minimalnej jest argumentowane kilkukrotną różnicą w~kwocie minimalnej pomiędzy Polską a krajami, które współtworzyły UE przed dołączeniem Polski. 
    
    Sukcesywne zwiększanie płacy minimalnej powinno mieć korzystny wpływ na warunki zatrudnienia, a~także redukcję stopnia ubóstwa w~kraju. Niestety, w~praktyce zwiększanie płacy minimalnej ma również poważne skutki uboczne:
    
    \begin{itemize}
        \item Koszty utrzymania pracownika wzrastają, co niekorzystnie wpływa na budżety małych i~średnich przedsiębiorców. Wzrost kosztów może oznaczać dla takich przedsiębiorców wybór pomiędzy redukcją zatrudnienia, a~,,zmuszeniem'' pracowników do pracy w~mniejszym wymiarze godzinowym lub przejścia na umowę-zlecenie albo umowę o~dzieło. Problem ten dotyka zwłaszcza regiony, w~których mała przedsiębiorczość nie rozwija się, co powoduje pogłębianie się recesji i~w~efekcie zmniejszaniu liczby miejsc pracy.
        \item Firmy, które uznają, że nowe stawki minimalne nie są warte pracy wykonywanej na poszczególnych stanowiskach dążą do agresywnej redukcji etatów. W~przypadku dużych przedsiębiorstw problemem również może być obniżenie standardu pracy poprzez przejście na inne, niekorzystne dla pracowników formy zatrudnienia.
    \end{itemize}

    W~rzeczywistości trudno jest zweryfikować wpływ wielkości płacy minimalnej na stopę bezrobocia. Historyczne przykłady (między innymi w~USA) pokazują, że zwiększenie kwoty minimalnej powoduje zwiększenie stopy bezrobocia. Natomiast wykres stopy bezrobocia w~Polsce pokazuje tendencję do stabilizacji odsetka osób niezatrudnionych, pomimo stałego wzrostu płacy minimalnej.
    
    \paragraph{Podwyższenie wieku emerytalnego}
    
    Na przełomie 2012 i~2013 roku Prezydent Bronisław Komorowski podpisał nowelizację ustawy emerytalnej, która podnosi docelowo wiek emerytalny dla kobiet oraz mężczyzn do 67 roku życia. Poprawka ta wzbudziła wiele kontrowersji i~była szeroko komentowana w~mediach. O~wpływie nowelizacji ustawy na bezrobocie wypowiadał się między innymi Premier Rzeczypospolitej Polskiej Donald Tusk, który w~wywiadzie dla TVN24 stwierdził, że nie ma związku pomiędzy wiekiem emerytalnym a wskaźnikiem bezrobocia~\cite{premier}.
    
    Zapominając na chwilę o~słowach Pana Premiera można z~łatwością wyobrazić sobie kilka scenariuszy, w~których podwyższony wiek emerytalny oddziałuje na odsetek bezrobotnych:
    
    \begin{itemize}
        \item Wzrost wieku emerytalnego może spowodować obniżenie składki, którą miesięcznie należy uregulować na poczet OFE. Sytuacja taka może spowodować, że zatrudnianie pracowników stanie się tańsze, a~przez to bardziej opłacalne niż do tej pory. Spowoduje to dynamiczny wzrost liczby ofert zatrudnienia na rynku pracy. Pracodawcy chętniej zatrudniać będą młodych, niedoświadczonych ludzi.
        \item Wzrost wieku emerytalnego nie wpłynie na liczbę ofert zatrudnienia na rynku pracy, spowoduje natomiast powiększenie konkurencji na jedno stanowisko. Zatrudnianie młodych, niedoświadczonych pracowników stanie się mniej opłacalne wobec dłuższego czasu pracy wykształconych specjalistów. Ponadto zabraknąć może miejsc dla pracowników, którzy z~racji na specyfikę wykonywanego zawodu nie będą w~stanie pozostać w~zawodzie do osiągnięcia wieku emerytalnego (np. pracownicy fizyczni). Dla tych osób przekwalifikowanie się będzie niezwykle trudne i~wobec braku możliwości wcześniejszego przejścia na emeryturę pozostaną oni bez możliwości zatrudnienia. 
    \end{itemize}
 
    W~praktyce trudno jednoznacznie odpowiedzieć na pytanie czy któryś z~powyższych czynników będzie miał dominujący wpływ na liczbę osób pozostających bez stałego zatrudnienia. Wydaje się, że wymienione powyżej okoliczności mogą wzajemnie znosić się co pozwala wyciągnąć wniosek, że podniesienie wieku emerytalnego nie wpłynie znacząco na stopę bezrobocia. Pokrywa się to zarówno ze słowami Pana Premiera, jak również z~wynikami badań zawartych w~pracy \emph{The Effects of Early Retirement on Youth Unemployment: The Case of Belgium}~\cite{imf}.
    
    \paragraph{Reforma urzędów pracy}
    
    Na przełomie zeszłego roku Sejm Rzeczypospolitej Polskiej zaakceptował nowelizację ustawy o~promocji zatrudnienia i~instytucjach rynku pracy~\cite{dz:u}. Głównym celem nowelizacji jest wprowadzenie profilowania osób zarejestrowanych jako bezrobotne w~urzędach pracy. Profilowanie ma obejmować podział bezrobotnych na trzy niezależne ,,koszyki'', które różnić się będą od siebie metodami udzielanej pomocy:
    
    \begin{description}
        \item[Osoby aktywne] Docelowo osoby aktywne jako pierwsze powinny otrzymywać będą oferty pracy. 
        \item[Osoby wymagające wsparcia] Głównym instrumentem pomocy są usługi oferowane przez urząd, takie jak staże czy szkolenia.
        \item[Osoby oddalone od rynku pracy] Wsparciem dla urzędu pracy podczas aktywizacji mają być ośrodki pomocy społecznej oraz agencje zatrudnienia.
    \end{description}

    Podział ten ma przyczynić się do eliminacji głównej wady aktywnych sposobów zwalczania bezrobocia - nieodpowiedniego alokowania funduszy ze względu na jednakowe traktowanie wszystkich bezrobotnych (przykład: organizowanie szkoleń dla osób niezainteresowanych poszerzaniem własnych kwalifikacji). 
    
    Nowelizacja może przyczynić się do wzrostu skuteczności aktywizacji osób bezrobotnych w~Polsce, przez co odsetek osób pozostających bez zatrudnienia obniży się. Powodzenie reformy zależy w~głównej mierze od sposobu jej realizacji. Jeżeli osoby bezrobotne zostaną przydzielone do grup w~odpowiedni sposób, cele nowelizacji zostaną najprawdopodobniej spełnione. 
    
    Niestety, dotychczasowe testy przeprowadzane przed wprowadzeniem reformy w życie każą poddawać w~wątpliwość sposób jej realizacji. Wyniki uzyskiwane w~poszczególnych urzędach pracy są od siebie bardzo odległe (w~jednym osób z~trzeciej grupy nie ma prawie w~ogóle, w~innym osoby te stanowią połowę wszystkich zarejestrowanych). Podział na grupy dokonywany będzie na podstawie formularza, co prawdopodobnie przyczyni się do wzrostu urzędowej biurokracji. Co więcej, sami urzędnicy mają problem z~ustaleniem odpowiednich pytań w~formularzu ze względu na kontrowersyjność i~subiektywność opisu niektórych cech, na przykład wyglądu zewnętrznego. Połączenie tych negatywnych doniesień na temat reformy ustawy każe przypuszczać, że nie wpłynie ona na niwelację stopy bezrobocia w~Polsce.

    
    
\begin{thebibliography}{9}
 \small
 
 \bibitem{ort}
  \emph{Wielki słownik ortograficzny},
  Polański E.,
  Wydawnictwo Naukowe PWN,
  Warszawa,
  2012,
  ISBN 978-83-01-16405-8.
 
 \bibitem{etym}
  \emph{Etymologiczny słownik języka polskiego, t.2}
  Bańkowski A.,
  Wydawnictwo Naukowe PWN,
  Warszawa,
  2000,
  ISBN 83-01-13017-2.
 
 \bibitem{sjp}
  \emph{Bezrobocie} [online],
  Słownik języka polskiego,
  http://sjp.pwn.pl/slownik/2444051/bezrobocie [dostęp: styczeń 2013].
 
 \bibitem{ust:pro:zat}
  \emph{Ustawa o promocji zatrudnienia i instytucjach rynku pracy.}
  Dz. U. z 2004 r. Nr 99, poz. 1001, Art. 2, punkt 2.
 
 \bibitem{eurostat}
  \emph{Employment and unemployment (LFS)} [online],
  Eurostat,
  http://epp.eurostat.ec.europa.eu/\\portal/page/portal/employment\_unemployment\_lfs/introduction [dostęp: styczeń 2013].
 
 \bibitem{eko}
  \emph{Ekonomia},
  Kamerschen D., McKenzie R., Nardinelli C.,
  Fundacja Gospodarcza NSZZ Solidarność,
  Gdańsk,
  1999,
  ISBN 83-00-03545-1, 83-00-03545-1.
  
 \bibitem{premier}
  \emph{Premier przekonywał, że wiek emerytalny nie wpływa na bezrobocie} [online],
  TVN 24,
  http://www.tvn24.pl/wideo/z-anteny/premier-przekonywal-ze-wiek-emerytalny-nie-wplywa-na-bezrobocie-tvn24,308255.html [dostęp: styczeń 2013].
 
 \bibitem{imf}
  \emph{The Effects of Early Retirement on Youth Unemployment: The Case of Belgium} [online],
  Jousten A., Lefèbvre M., Perelman S., Pestieau P.,
  International Monetary Fund,
  2008,
  https://www.imf.org/external/pubs/ft/wp/2008/wp0830.pdf  [dostęp: styczeń 2013].
 
 \bibitem{dz:u}
  Dz.U. z 2013 r. poz. 674 z późn. zm.
  
\end{thebibliography}
    
\end{document}
